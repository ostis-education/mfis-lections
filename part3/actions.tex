\title{Лекция 18\\Представление в базе знаний процессов и действий}
\author[]{Шункевич Д.В.}
\institute[]{Белорусский государственный университет информатики и радиоэлектроники}

\begin{frame}
	\titlepage
\end{frame}

\begin{frame}{\\Содержание лекции}
	\topline
	\justifying
	Понятие процесса, воздействия, действия. Описание действий в семантической памяти, элементарные события в семантической памяти. Классификация действий, в том числе выполняемых в семантической памяти. Отношения, заданные на множестве действий. Контекст выполнения действия. Начальная и конечная ситуация процесса. 
\end{frame}

\begin{frame}{\large Понятие процесса}
\vspace{20}
    Процессом называется процесс, связанный с преобразованием некоторой временной сущности из одного состояния в другое (с переходом от одной ситуации к другой).\\
    Каждый \textbf{\textit{процесс}} определяется возникновением или исчезновением связей, связывающих как заданную \textit{временную сущность} с другими сущностями, так и части указанной \textit{временной сущности} с другими сущностями.\\
    Многим \textit{\textbf{процессам}} можно поставить в соответствие \textit{ситуацию}, которая является его \textit{начальной ситуацией*} и ситуацию, являющейся его \textit{конечной ситуацией*} (указать ситуации, переход между которыми осуществляется в ходе процесса).\\
    \textit{Процесс} представляет собой изменяющуюся во времени динамическую структуру, полностью представить его в БЗ в общем случае невозможно. Однако, можно ввести sc-элемент, обозначающий конкретный процесс с описанием его декомпозиции на более частные, или описать ситуации, соответствующие его состоянием в разные моменты времени.
\end{frame}

\begin{frame}{\large Понятие воздействия}
\vspace{23}
    Воздействие - процесс, в котором могут быть явно выделены хотя бы одна воздействующая сущность (субъект') и хотя бы одна сущность, на которую осуществляется воздействие (объект').\\
    Субъект' - сущность, создающая причину изменения другой сущности.\\
    Объект' - сущность, являющаяся в рамках заданного воздействия исходным условием (аргументом), необходимым для выполнения этого воздействия.\\
    Воздействия являются частным видом процессов и ими наследуются все свойства процессов. В частности, используются все параметры, заданные на множестве процессов, например, длительность*, момент начала процесса*, момент завершения процесса^.\\
    Так как воздействие является процессом и представляет собой динамическую структуру, то и знак субъекта воздействия′ , и знак объекта воздействия ′ являются элементами данной структуры. В связи с этим можно рассматривать отношения субъект воздействия ′ и объект воздействия ′ как ролевые отношения. Данный факт не запрещает вводить аналогичные неролевые отношения, однако это нецелесообразно.
\end{frame}

\begin{frame}{\large Понятие действия}
\vspace{23}
    действие - воздействие, в котором субъект' осуществляет воздействие целенаправленно, то есть в соответствии с некоторой целью*, с возможным применением некоторых инструментов.\\
    Каждое действие, выполняемое тем или иным субъектом, трактуется как процесс решения некоторой задачи, то есть процесс достижения заданной цели* в заданных условиях, и, следовательно, выполняется целенаправленно. При этом явное указание действия и его связи с конкретной задачей может не всегда присутствовать в памяти. Некоторые задачи могут решаться определенными субъектами перманентно, например, оптимизация базы знаний, поиск некорректностей и так далее и для подобных задач не всегда есть необходимость явно вводить структуру, являющуюся формулировкой задачи.\\ 
    Каждое действие может обозначать сколь угодно малое преобразование, осуществляемое во внешней среде либо в памяти некоторой кибернетической системы, однако в памяти явно вводятся знаки только тех действий, для которых есть необходимость явно хранить в памяти их спецификацию в течение некоторого времени.
\end{frame}

\begin{frame}{\large Элементарные события в семантической памяти}
\vspace{25}
    Событием называется процесс, длительность которого в данном контексте считается несущественной.\\
    Элементарным событием в sc-памяти называется такое событие, в результате выполнения которого изменяется состояние только одного sc-элемента.
    Элементарные события в sc-памяти:
    \begin{textitemize}
        \item событие добавления sc-дуги, выходящей из заданного sc-элемента
        \item событие добавления sc-дуги, входящей в заданный sc-элемент
        \item событие добавления sc-ребра, инцидентного заданному sc-элементу
        \item событие удаления sc-дуги, выходящей из заданного sc-элемента
        \item событие удаления sc-дуги, входящей в заданный sc-элемент
        \item событие удаления sc-ребра, инцидентного заданному sc-элементу
        \item событие удаления sc-элемента
        \item событие изменения содержимого файла
    \end{textitemize}
\end{frame}

\begin{frame}{Классификация действий}
    Класс действий имеет разбиение по следующим признакам:
\begin{textitemize}
	\item место выполнения действия;
	\item функциональная сложность действия;
	\item многоагентность выполнения действия;
	\item текущее состояние действия.
\end{textitemize}
Далее рассмотрим разбиение по каждому признаку по отдельности
\end{frame}

\begin{frame}{Действия по признаку места выполнения}
    \begin{textitemize}
        \item \textbf{информационное действие}. Действие, направленное на обработку информации, хранимой в памяти кибернетической системы. Результатом выполнения является некоторое новое состояние памяти системы, достигнутое исключительно путём преобразования информации, хранящейся в памяти системы.
        \item \textbf{поведенческое действие}. Действие, направленное на изменение состояния внешней среды. Под внешней средой понимаются и компоненты системы, внешние с точки зрения памяти, не являющиеся хранимыми в ней информационными конструкциями. К ним можно отнести различные манипуляторы, средства воздействия системы на внешний мир. К поведенческим задачам можно отнести изменение состояния механической конечности робота, вывод информации на экран и другое.
    \end{textitemize}
\end{frame}
\begin{frame}{}
\begin{textitemize}
    \item \textbf{рецепторное действие}. Каждое рецепторное действие обозначает класс действий, которые осуществляют преобразования в памяти субъекта действия под воздействием внешней среды.
        \item \textbf{эффекторное действие}. Каждое эффекторное действие обозначает класс действий, которые осуществляют преобразования внешней среды под воздействием памяти субъекта воздействия.
\end{textitemize}
\end{frame}

\begin{frame}{\large Действия по признаку функциональной сложности}
\vspace{20}
    \begin{textitemize}
        \item \textbf{атомарное действие}. Выполняется одним индивидуальным субъектом и не включает в себя выполнение каких-либо дочерних действий.
        \item \textbf{неатомарное легко выполнимое действие}. Для выполнения этого действия известен соответствующий метод и соответствующие этому методу исходные данные, имеются в наличии все необходимые исходные объекты и все средства.
        \item \textbf{неатомарное трудно выполнимое действие}. Для выполнения в текущий момент либо неизвестен соответствующий метод, либо возможные методы известны, но отсутствуют условия их применения. Это действие декомпозируется на несколько самостоятельных поддействий, каждое из которых выявляет противоречия конкретного формализуемого вида, для которого в базе знаний существует точное определение. 
    \end{textitemize}
\end{frame}
\small{Неатомарное действие - действие, выполнение которого требует декомпозиции этого действия на множество поддействий. Более частные действия могу выполняться как последовательно, так и параллельно. \\
Декомпозиция неатомарного действия на поддействия может иметь весьма сложный иерархический вид с большим числом уровней иерархии (поддействиями неатомарного действия могут быть также неатомарные действия).\\
Уровень сложности действия можно определять общим числом его поддействий и числом уровней иерархии этих поддействий. Примером может служить запись одной и той же процедурной программы на языке программирования более высокого уровня и на языке программирования более низкого уровня.\\
Темпоральные соотношения между поддействиями неатомарного действия могут быть самые различные, но в пройстейшем случае неатомарное действие представляет собой строгую последовательность действий более низкого уровня иерархии.\\
В состав неатомарного действия могут входить не только поддействия этого неатомарного действия, но и специальные поддействия, осуществляющие управление процессом выполнения неатомарного действия, и, в частности, поддействия, осуществляющие инициирование поддействий, передачу управления поддействиям.}

\begin{frame}{Действия по признаку количества субъектов, выполняющих действие}
    \begin{textitemize}
        \item \textbf{индивидуальное действие}.
        \item \textbf{коллективное действие}.
    \end{textitemize}
    Действия могут выполняться как агентом или группой агентов, так и кибернетической системой или их группой
\end{frame}

\begin{frame}{\large Действия по признаку текущего состояния действия}
\vspace{20}
    \begin{textitemize}
        \item \textbf{планируемое действие}. Действия, начало выполнения которых запланировано на какой-либо момент в будущем. 
        \item \textbf{инициированное действие}. Действия, выполнение которых инициировано в результате какого-либо события. Принадлежность некоторого действия множеству инициированных действий говорит о том, что, по мнению некоторого субъекта оно готово к выполнению и должно быть выполнено.  В общем случае действия могут быть инициированы по следующим причинам:
        \begin{textitemize}
            \item явно путём проведения соответствующей sc-дуги принадлежности каким-либо субъектом. Может быть инициировано как внутренним sc-агентом, так и пользователем при помощи интерфейса.
            \item в результате того, что одно или несколько действий, предшествовавших данному в рамках некоторой декомпозиции, стали прошлыми сущностями.
            \item в результате того, что в памяти системы появилась конструкция, соответствующая некоторому условию инициирования sc-агента, который должен выполнить данное действие.
        \end{textitemize}
    \end{textitemize}
\end{frame}
\begin{frame}{}
    \begin{textitemize}
        \item \textbf{выполняемое действие}. Во множество выполняемых действий входят действия, к выполнению которых приступил какой-либо из соответствующих субъектов. Наличие действия в данном множестве говорит о том, что рассматриваемое действие уже попало во множество инициированных действий, а так же существует как минимум один субъект, условие инициирования которого соответствует спецификации данного действия. После того, как процесс выполнения завершился, действие должно быть удалено из множества выполняемых действий и добавлено во множество выполненных действий. 
        \item \textbf{прерванное действие}. В данное множество входят действия, которые были инициированы, но их выполнение по каким-либо причинам невозможно.
        \item \textbf{успешно выполненное действие}.
        \item \textbf{безуспешно выполненное действие}.
        \item \textbf{отменённое действие}.
    \end{textitemize}
\end{frame}
\begin{frame}{}
    Во множество выполненных действий попадают действия, выполнение которых завершено с точки зрения субъекта, осуществлявшего их выполнение. Понятие выполненного действия является относительным, поскольку разные субъекты могут считать одно и то же действие выполненным или выполняющимся.\\
    Во множество успешно выполненных действий попадают действия, выполнение которых завершено с точки зрения субъекта, осуществлявшего их выполнение, то есть выполнена поставленная цель, например получены решение и ответ какой-либо задачи. Очень важно отметить, что в общем случае выделить критерии успешности или безуспешности выполнения действий того или иного класса невозможно, поскольку эти критерии, во-первых, зависят от контекста, во-вторых, могут быть разными с точки зрения разных субъектов. Однозначно критерии успешности выполнения действий могут быть сформулированы для некоторых частных классов действий, таким образом понятие успешно выполненное действие является относительным.\\
    Если действие было выполнено успешно, то, в случае действия по генерации каких-либо знаний, к действию при помощи связки отношения результат* приписывается sc-конструкция, описывающая результат выполнения указанного действия. В случае, когда действие направлено на какие-либо изменения базы знаний, sc-конструкция, описывающая результат действия, формируется в соответствии с правилами описания истории изменений базы знаний.
\end{frame}
\begin{frame}{}
    В случае, когда успешное выполнение действия приводит к изменению какой-либо конструкции в sc-памяти, которое необходимо занести в историю изменений базы знаний или использовать для демонстрации протокола решения задачи, то генерируется соответствующая связка отношения результат*, связывающая задачу и sc-конструкцию, описывающую данное изменение.\\
Во множество безуспешно выполненных действий попадают действия, выполнение которых не было успешно завершено с точки зрения субъекта, осуществлявшего их выполнение, по каким-либо причинам.
Две основные причины, по которым складывается данная ситуация:
\begin{textitemize}
    \item соответствующая задача сформулирована некорректно
    \item формулировка соответствующей задачи корректна и понятна системе, однако решение данной задачи в текущий момент не может быть получено за удовлетворительные с точки зрения заказчика или исполнителя сроки
\end{textitemize}
\end{frame}
\begin{frame}{}
Для конкретизации факта некорректности формулировки задачи можно выделить ряд более частных классов безуспешно выполненных действий, например:
\begin{textitemize}
    \item действие, спецификация которого противоречит другим знаниям системы (например, не выполняется неравенство треугольника)
    \item действие, при спецификации которого использованы понятия, неизвестные системе
    \item действие, выполнение которого невозможно из-за недостаточности данных (например, найти площадь треугольника по двум сторонам)
\end{textitemize}
    Для конкретизации факта безуспешности выполнения некоторого действия в системе могут также использоваться дополнительные подмножества данного множества, при необходимости снабженные естественно-языковыми или формальными комментариями.\\
Во множество действий, выполненных с ошибкой, попадают действия, выполнение которых не было успешно завершено с точки зрения субъекта, осуществлявшего их выполнение, по причине возникновения какой-либо ошибки, например, некорректности спецификации данного действия или нарушения ее целостности каким-либо субъектом (в случае действия в sc-памяти).
\end{frame}

\begin{frame}{Отношения, заданные на множестве действий}
    На множестве действий можно задать некоторые отношения:
    \begin{textitemize}
        \item поддействие* - связывает \textit{действие} и более простое частное \textit{действие}, выполнение которого необходимо для выполнения исходного более общего \textit{действия}.
        \item наддействие* - связывает \textit{действие} и другое \textit{действие}, являющееся частью заданного \textit{действия} более высокого уровня иерархии*.
        \item непосредственное поддействие* - связывает \textit{действие} и \textit{поддействие} заданного \textit{действия}, для которого не существует \textit{наддействия}, которое было бы также \textit{поддействием} заданного \textit{действия*}.
    \end{textitemize}
\end{frame}

\begin{frame}{\large Контекст выполнения действия}
\vspace{30}
    Связки отношения \textbf{контекст действия*} связывают sc-элементы, обозначающие действие и структуры, обозначающие контекст выполнения данного действия, то есть некоторую дополнительную информации о тех сущностях, которые входят в описание \textit{цели*}. Как правило, контекст используется для указания собственно условия некоторой задачи, того, что дано, т.е. тех знаний, которые можно использовать для вывода новых знаний при решении задачи. Таким образом, контекст непосредственно влияет на то, как будет решаться та или иная задача, при этом даже задачи соответствующие одному классу действий, могут решаться по-разному.\\
    Первым компонентом связок отношения \textbf{контекст действия*} является знак \textit{действия}, вторым - знак \textit{структуры}, обозначающей контекст.\\
    Контекст может быть представлен не только в виде атомарного фактографического высказывания, но и в виде высказывания более сложного вида, например:
    \begin{textitemize}
        \item Определение множества, используемого в описании \textit{цели*}.
        \item Утверждение, учёт которого может быть полезен в решении задач.
    \end{textitemize}
\end{frame}

\begin{frame}{\large Исходная, начальная и конечная ситуация}
\vspace{20}
Связки отношения \textit{\textbf{исходная ситуация*}} связывают некоторый \textit{процесс} и некоторую ситуацию, являющуюся начальной для этого \textit{процесса}, и, как правило, изменяемой в течение выполнения этого \textit{процесса}.
    Первым компонентом каждой связки отношения \textit{исходная ситуация*} является знак \textit{процесса}, вторым – знак начальной \textit{ситуации.}\\
    \textit{исходная ситуация*} действия содержит описание тех исходных условий, которых оказалось достаточно для инициирования данного действия, в то время как \textit{начальная ситуация*} является более общим понятием и в общем случае не предполагает такого описания.\\
    Связки отношения \textit{\textbf{конечная ситуация*}} связывают некоторый процесс и некоторую ситуацию, ставшую результатом выполнения этого процесса, то есть его следствием. Конечная ситуация не всегда совпадает с ситуацией, рассматриваемой как целевая.
Первым компонентом каждой связки отношения \textit{конечная ситуация*} является знак \textit{процесса}, вторым – знак конечной \textit{ситуации}.
\end{frame}
