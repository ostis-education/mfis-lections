\title{Лекция 23\\Принципы решения задач в интеллектуальных компьютерных системах нового поколения}
\author[]{Шункевич Д.В.}
\institute[]{Белорусский государственный университет информатики и радиоэлектроники}

\begin{frame}
	\titlepage
\end{frame}

\begin{frame}{\\Содержание лекции}
	\topline
	\justifying
	Стратегии решения задач в компьютерных системах нового поколения. Основные модели логического вывода. Принципы организации логического вывода в интеллектуальных компьютерных системах нового поколения. Принципы реализации параллельной обработки знаний в семантической памяти. Связь с базовым языком программирования для обработки баз знаний. 
\end{frame}

\begin{frame}{\\Стратегии решения задач}
    \begin{scn}
    В литературе, посвященной построению решателей задач, встречается понятие стратегии решения задач. Определим его как метаметод решения задач, обеспечивающий либо поиск одного релевантного известного метода, либо синтез целенаправленной последовательности акций применения в общем случае различных известных методов. Можно говорить об универсальном метаметоде решения задач, объясняющем всевозможные частные стратегии. В частности, можно говорить о нескольких глобальных стратегиях решения информационных задач в базах знаний.
    \end{scn}
\end{frame}

\begin{frame}{\\Стратегии решения задач}
    \begin{scn}
Пусть в базе знаний появился знак инициированного действия с формулировкой, соответствующей цели, направленной только на изменение состояния базы знаний. И пусть текущее состояние базы знаний не содержит контекст (исходные данные), достаточный для достижения указанной выше цели, т.е. такой контекст, для которого в доступном наборе методов имеется метод, использование которого позволяет достичь указанной выше цели.
    \end{scn}
\end{frame}

\begin{frame}{\\Основные стратегии решения задач}
    \vspace{10mm}
    \begin{scn}
        Для решения задачи, когда исходных данных недостаточно, существует 3 основные подхода:
        \begin{textitemize}
            \item декомпозиция (сведение изначальной цели к иерархической системе и/или подцелей (и/или подзадач) на основе анализа текущего состояния базы знаний и анализа того, чего не хватает в базе знаний для использования ого или иного метода). Формализация понятий действия, задачи, метода, средства, навыка и технологии. При этом наибольшее внимание уделяется методам, для создания условий использования которых требуется меньше усилий. В конечном счете мы должны дойти (на самом нижнем уровне иерархии) до подцелей, контекст которых достаточен для применения одного из имеющихся методов (программ) решения задач;
        \end{textitemize}
    \end{scn}
\end{frame}

\begin{frame}{\\Основные стратегии решения задач}
    \begin{scn}
        \begin{textitemize}
            \item генерация новых знаний в семантической окрестности формулировки изначальной цели с помощью любых доступных методов в надежде получить такое состояние базы знаний, которое будет содержать нужный контекст (достаточные исходные данные) для достижения изначальной цели с помощью какого-либо имеющегося метода решения задач;
            \item комбинация первого и второго подходов.
        \end{textitemize}
    \end{scn}
\end{frame}

\begin{frame}{\\Задачи логики}
    \begin{scn}
        Логика решает задачи доказательства истинности высказываний, аргументации того или иного высказывания, задачу генерации и опровержения гипотез. Некоторые гипотезы могут быть опровергнуты, однако извлекая причины того, почему гипотеза опровергнута, можно изменить посылку гипотезы так, чтобы создать новую гипотезу, которая впоследствии может стать теоремой.
    \end{scn}
\end{frame}

\begin{frame}{\\Логический вывод}
    \vspace{10mm}
    \begin{scn}
        Предметная область логических формул, высказываний и формальных теорий задаёт денотационную семантику логических формул, высказываний и формальных теорий и содержит формальную спецификацию понятий, необходимых для формирования логических формул и высказываний любых логик, в том числе традиционных, нечётких, правдоподобных, темпоральных, логик умолчания и любых других. Логические формулы и высказывания интерпретируются с помощью понятий, описанных в Предметной области логических моделей решения задач, включающую модель и реализацию абстрактных агентов, необходимых для решения логических задач. Эта предметная область включает в себя спецификацию таких понятий, как логический вывод, правила вывода, равносильные преобразования и аксиомные схемы.
    \end{scn}
\end{frame}

\begin{frame}{\\Формальные языки}
    \begin{scn}
        Современная логика изучает формальные языки, служащие для выражения логических рассуждений. Логический язык — формальный язык, предназначенный для воспроизведения логических форм контекстов естественного языка, а также выражения логических законов и способов правильных рассуждений в логических теориях, строящихся в данном языке.
        \\Язык SCL — подъязык SC-кода для записи логических утверждений. Над высказываниями языка SCL можно проводить логический вывод.
    \end{scn}
\end{frame}

\begin{frame}{\\Вывод в формальной системе}
    \begin{scn}
    \vspace{10mm}
        Выводом в формальной системе называется любая последовательность формул, где любая формула либо аксиома этой формальной системы, либо непосредственное следствие каких-либо предыдущих формул по одному из правил вывода. Идея выводимости центральна в логике: в любой формальной аксиоматической теории ‘теорема’ – это формула, которая выводится из аксиом. Правильность умозаключений вводится и проверяется совершенно формально, без какой-либо связи с истинностью входящих в него посылок, т.е. исключительно с точки зрения структуры рассуждения. Если нам удалось доказать, пользуясь методами формальной логики, правильность рассуждения, и нам известно из опыта, что все используемые посылки истинны, то мы можем быть уверены в истинности заключения. Истинность используемых посылок задаётся состоянием базы знаний.
    \end{scn}
\end{frame}

\begin{frame}{\\Логические методы решения задач}
    \vspace{10mm}
    \begin{scn}
    \begin{textitemize}
        \item Классический дедуктивный вывод. Наиболее популярный метод при построении автоматических решателей задач, так как всегда дает достоверный результат. Дедуктивный вывод включает в себя прямой и обратный логический вывод (принцип резолюции, процедуру Эрбрана и др.), все виды силлогизмов и т.д. Основной проблемой дедуктивного вывода является невозможность его использования в ряде случаев, когда отсутствуют достоверные знания. (Sethy S.S.MediaIS-2021art, Averin A.I..UsingPiDI-2004art)
        \item Индуктивный вывод. Предоставляет возможность в процессе решения использовать различные предположения, что делает его удобным для использования в слабоформализованных и трудноформализуемых предметных областях, например при построении систем медицинской диагностики. (Norton J.D..aDemonstrationotIoCoII-2019art, YuxuanZ..MissiEAKGIItDGLaT-2022art)
    \end{textitemize}
    \end{scn}
\end{frame}
\begin{frame}{\\Логические методы решения задач}
    \vspace{12mm}
    \begin{scn}
    \begin{textitemize}
        \item  Абдуктивный вывод. Под абдуктивным выводом в искусственном интеллекте, как правило, понимается вывод наилучшего абдуктивного объяснения, т.е. объяснения некоторого события, ставшего неожиданным для системы. Причем наилучшим считается такое объяснение, которое удовлетворяет специальным критериям, определяемым в зависимости от решаемой задачи и используемой формализации. (Safawi A.R..tDecisPoAI-2015art, Gungov A.tAmpliLiDtAoAIiCR-2018art)
        \item Нечеткая логика. Теория нечетких множеств и, соответственно, нечетких логик, также применяется в системах, связанных с трудноформализуемыми предметными областями. Здесь импликативные высказывания могут рассматриваться как "если истинна посылка, то с некоторой вероятностью (часто или редко) истинно заключение", в отличие от классической логики, где зачастую используются статические предметные области и выражение "часто или редко" не применимо. (Uehara K..FuzzyIIPaP-2017art, Geramian A..FuzzyISAfFAiAI-2017art, Son L.H..PictuISaNFISoPFS-2017art)
    \end{textitemize}
    \end{scn}
\end{frame}
\begin{frame}{\\Логические методы решения задач}
    \vspace{10mm}
    \begin{scn}
    \begin{textitemize}
        \item Логика умолчаний. Логика умолчаний применяется, в том числе, для того, чтобы оптимизировать процесс рассуждений, дополняя процесс достоверного вывода вероятностными предположениями в тех случаях, когда вероятность ошибки крайне мала. (Lupea M.aTheorPfCaRDL2002art, Weydert E.DefauLaPaTP-2022art)
        \item Темпоральная логика. Её применение является очень актуальным для нестатичных предметных областей, в которых истинность того или иного утверждения меняется со временем, что существенно влияет на ход решения какой-либо задачи. (Chen G..TempoLIfFDoSSwGPD-2021art, Рыбаков В.В.МультВНЛЛПД-2020ст)        
    \end{textitemize}
    \end{scn}   
\end{frame}


\begin{frame}{\\Принципы организации логического вывода в интеллектуальных компьютерных системах нового поколения}
    \vspace{10mm}
    \begin{scn}
        Технология OSTIS позволяет интегрировать любые модели решения задач и принципы логического вывода для решения задач в интеллектуальных системах на основе общей формальной модели. Для того, чтобы использовать какую-либо новую или существующую модель, необходимо привести ее к предлагаемому формализму, что позволит интегрировать и синхронизировать ее с уже имеющимися в соответствующей библиотеке совместимых компонентов. Формализм SC-кода позволяет описывать отношения между понятиями любой формы и сложности, что делает его подходящим вариантом для использования логического вывода в интеллектуальных компьютерных системах нового поколения. А также воспользоваться техникой иерархии за счёт онтологического подхода, лежащего в основе баз знаний ostis-систем.
    \end{scn} 
\end{frame}

\begin{frame}{\\Принципы организации логического вывода в интеллектуальных компьютерных системах нового поколения}
    \begin{scn}
        Наследование предметных областей позволяет использовать описанные логики и их компоненты при описании любых логик. Базовые понятия позволяют разработчикам интеллектуальной системы добавлять новые логики. Для реализации конкретной логической модели решения задач необходимо создать предметную область, которая будет дочерней по отношению к Предметной области логических моделей решения задач и предметной области некоторого логического языка, например, языка логики высказываний, языка логики предикатов, языка нечёткой логики и других.
    \end{scn} 
\end{frame}

\begin{frame}{\\Абстрактный SC-агент}
\vspace{12mm}
\begin{scn}
Абстрактная scl-машина является машиной логического вывода и относится к классу абстрактных sc-машин. Внутренним языком scl-машины является графовый логический язык SCL, её операции соответствуют правилам логического вывода. Семейство специализированных абстрактных графодинамических машин обработки знаний является формальным уточнением операционной семантики указанных выше специализированных графовых языков представления знаний, каждому из которых соответствует одна или несколько абстрактных машин. Эти абстрактные машины соответствуют различным моделям решения задач, логикам, моделям правдоподобных рассуждений. Агент из семейства агентов логического вывода может представлять собой какое-либо правило вывода, применяемое для решения логической задачи. Кроме того, необходимы агенты для выполнения равносильных преобразований логической формулы (например, записать формулу эквиваленции как конъюнкцию двух дизъюнкций) и другие агенты, помогающие применять правила вывода на множестве формул языка логики.
\end{scn}
\end{frame}

\begin{frame}{\\Виды абстрактных SC-Агентов}
    \begin{scn}
    \scnheader{Абстрактная scl-машина}
		\begin{scnrelfromset}{декомпозиция абстрактного sc-агента*}
			\scnitem{Абстрактный sc-агент применения правила вывода}
            \scnitem{Абстрактный sc-агент эквивалентных преобразований логической формулы}
            \scnitem{Абстрактный sc-агент прямого логического вывода}
			\scnitem{Абстрактный sc-агент обратного логического вывода}
		\end{scnrelfromset}	
    \end{scn}
\end{frame}

\begin{frame}{\Large Абстрактный sc-агент применения правила вывода}
\vspace{8mm}
  Абстрактный sc-агент применения правила вывода применяет заданное правило вывода с заданными логическими формулами. Агент активируется при появлении в sc-памяти инициированного действия, принадлежащего классу действие применение правила вывода. После проверки sc-агентом условия инициирования выполняется процесс применения правила вывода, который заключается в проверке, существует ли в sc-памяти структуры, соответствующие условию применения данного правила и генерации sc-конструкций в соответствии с применяемым правилом. Агент применения правила вывода зачастую используется в процессе работы агентов прямого логического вывода, обратного логического вывода и других агентов.  
\end{frame}

\begin{frame}{\\Пример абстрактного sc-агента применения правила вывода}
%    подтянуть картиночку 326стр монография
\end{frame}

\begin{frame}{\Large Абстрактный sc-агент эквивалентных преобразований логической формулы}
\vspace{10mm}
  Абстрактный sc-агент эквивалентных преобразований логической формулы применяет некоторые правила, которые приводят логическую формулу в определённый вид. Агент активируется при появлении в sc-памяти инициированного действия, принадлежащего соответствующему классу действий. После проверки sc-агентом условия инициирования выполняется процесс преобразования формулы из одной формы в другую, при этом никакие новые знания в sc-памяти с точки зрения исследуемой ПрО не генерируются. Ответом данного агента является множество формул, эквивалентных по смыслу, но различных по форме представления. Такими формами могут быть, например, КНФ или ДНФ. Агент эквивалентных преобразований зачастую вызывается в процессе работы агента применения правила вывода, так как логические формулы не всегда находятся в той форме, которая доступна для применения того или иного правила вывода, однако может быть приведена к нужной форме.  
\end{frame}

\begin{frame}{\Large Абстрактный sc-агент прямого логического вывода}
\vspace{10mm}
  Абстрактный sc-агент прямого логического вывода генерирует новые знания на основе некоторых логических утверждений. Агент активируется при появлении в sc-памяти инициированного действия, принадлежащего соответствующему классу действий. После проверки sc-агентом условия инициирования выполняется процесс прямого логического вывода, состоящий из циклических операций применения правил вывода, генерации новых знаний в sc-памяти и проверки некоторого условия, например, появление в памяти sc-элементов из целевой sc-структуры. Агент принимает целевую структуру, множество формул, используемых в ходе вывода агентом применения правил вывода, мн-во правил вывода, входную и выходную структуры. В результате выполнения действия, в sc-памяти формируется sc-структура, представляющая собой дерево решения. Это дерево состоит из последовательности узлов, представляющих собой применённые правила, что привели к появлению в sc-памяти требуемых знаний (может быть пустым, если требуемую структуру не удалось сгенерировать в ходе логического вывода).  
\end{frame}

\begin{frame}{\\Пример абстрактного sc-агента применения прямого логического вывода}
%    подтянуть картиночку 327стр монография
\end{frame}

\begin{frame}{\Large Абстрактный sc-агент обратного логического вывода}
\vspace{8mm}
  Абстрактный sc-агент обратного логического вывода проверяет гипотезы. Некоторые гипотезы могут быть опровергнуты, однако извлекая причины того, почему гипотеза опровергнута, можно изменить посылку гипотезы так, чтобы создать новую гипотезу, которая впоследствии может стать полезной теоремой. Данный sc-агент активируется при появлении в sc-памяти инициированного действия, принадлежащего классу действие обратного логического вывода. После проверки sc-агентом условия инициирования выполняется процесс обратного логического вывода, который схож с процессом прямого логического вывода за исключением того, что поиск правил основывается не на посылках формул, а на их следствиях. Ответом данного агента будет также дерево вывода, которое показывает, с использованием каких правил можно доказать или опровергнуть выдвинутую гипотезу. 
\end{frame}


