\title{Лекция 20\\Представление в базе знаний методов и навыков}
\author[]{Шункевич Д.В.}
\institute[]{Белорусский государственный университет информатики и радиоэлектроники}

\begin{frame}
	\titlepage
\end{frame}

\begin{frame}{\\Содержание лекции}
	\topline
	\justifying
	Понятие метода, классификация методов. Отношения, заданные на множестве методов. Денотационная и операционная семантика метода. Примеры описания в базе знаний. Понятие языка описания методов. Пример описания процедурного метода в базе знаний. Статические и динамические аргументы действий. Пример описания метода с зависимостями между поддействиями. Понятие стратегии решения задач. Понятие модели решения задач. Понятие навыка, классификация навыков. Связь с базовым языком программирования для обработки баз знаний.
\end{frame}

\begin{frame}{\\Понятие метода}
	\topline
	\justifying
    
    Под методом будем понимать описание того, как может быть выполнено любое или почти любое (с явным указанием исключений) действие,
    принадлежащее соответствующему классу действий.
    
    \begin{SCn}
    \scnheader{метод}
    \scnrelto{второй домен}{метод*}
    \scnidtf{обобщенная спецификация выполнения действий соответствующего класса}
    \scnidtf{обобщенная спецификация решения задач соответствующего класса}
    \end{SCn}
\end{frame}

\begin{frame}{\\Понятие метода}
	\topline
	\justifying
    \begin{SCn}
    \scnheader{метод}
    \scnidtf{знание о том, как можно решать задачи соответствующего класса}
    \scnsubset{знание}
    \scniselement{вид знаний}
    \scnidtf{способ}
    \scnidtf{\textbf{программа}}
    \scnidtf{знание о том, как надо решать задачи соответствующего класса задач (множества эквивалентных (однотипных, похожих) задач)}
    \scnidtf{метод (способ) решения некоторого (соответствующего) класса задач}
    \scnidtf{информация (знание), достаточная для того, чтобы решить любую \textit{задачу}, принадлежащую соответствующему \textit{классу задач}, с помощью соответствующей \textit{модели решения задач}}
    \end{SCn}
\end{frame}

\begin{frame}{\\Классификация методов}
	\topline
	\justifying
 
    \vspace{10mm}
    
    Как действия и задачи, методы могут быть классифицированы по различным классам. Будем называть \textbf{\textit{классом методов}} множество методов, для которых можно унифицировать представление (спецификацию) этих методов.
    \begin{SCn}
    
    \scnheader{класс методов}
    \scnrelto{семейство подклассов}{метод}
    \scnidtf{множество методов решений задач, которому соответствует специальный язык (например, sc-язык), обеспечивающий представление методов из этого множества}
    \scnidtf{множество методов, которому ставится в соответствие отдельная модель решения задач}
    \end{SCn}
\end{frame}

\begin{frame}{\\Классификация методов}
	\topline
	\justifying
     \begin{SCn}
     \scnheader{класс методов}
    \scnhaselement{процедурный метод решения задач}
    \scnhaselement{декларативный метод решения задач}
    \scnhaselement{искусственная нейронная сеть}
    \scnhaselement{генетический "алгоритм"{}}
    \end{SCn}
\end{frame}

\begin{frame}{\\Классификация методов}
	\topline
	\justifying
     \begin{SCn}
	\scnheader{процедурный метод решения задач}
		\scnsuperset{алгоритмический метод решения задач}
    \scnheader{декларативный метод решения задач}
		\scnsuperset{продукционный метод решения задач}
		\scnsuperset{функциональный метод решения задач}
		\scnsuperset{логический метод решения задач}
    \scnheader{искусственная нейронная сеть}
		\scnidtf{класс методов решения задач на основе искусственных нейронных сетей}
  \end{SCn}
\end{frame}

\begin{frame}{\\Отношения, заданные на множестве методов}
	\topline
	\justifying
	\begin{SCn}
    \scnheader{отношение, заданное на множестве методов}
    \scnhaselement{подметод*}
    \scnheader{подметод*}
    	\scnidtf{подпрограмма*}
    	\scnidtf{быть методом, использование которого (обращение к которому) предполагается при реализации заданного метода*}
\end{SCn}
\end{frame}

\begin{frame}{\\Операционная семантика метода}
	\topline
	\justifying
	\begin{SCn}
    \scnheader{операционная семантика метода*}
    \scnsubset{спецификация*}
    \scnidtf{семейство методов, обеспечивающих интерпретацию заданного метода*}
    \scnidtf{формальное описание интерпретатора заданного метода*}
    \scnrelfrom{второй домен}{\textbf{операционная семантика метода}}
    \end{SCn}
\end{frame}

\begin{frame}{\\Операционная семантика метода}
	\topline
	\justifying
	\begin{SCn}
    \scnheader{операционная семантика метода}
    \scnsuperset{\textbf{полное представление операционной семантики метода}}
    \scnidtf{представление \textit{операционной семантики метода}, доведенное (детализированное) до уровня всех \textit{спецификаций элементарных действий}, выполняемых в процессе интерпретации соответствующего \textit{метода}}
    \scnheader{полное представление операционной семантики метода}
    \scnidtf{представление \textit{операционной семантики метода}, доведенное (детализированное) до уровня всех \textit{спецификаций элементарных действий}, выполняемых в процессе интерпретации соответствующего \textit{метода}}
    \end{SCn}
\end{frame}

\begin{frame}{\\Денотационная семантика метода}
	\topline
	\justifying
 \begin{SCn}
     \scnheader{декларативная семантика метода*}
    \scnsubset{спецификация*}
    \scnidtf{описание системы понятий, которые используются в рамках данного метода*}
 \end{SCn}
\end{frame}

% \begin{frame}{\\Примеры описания в базе знаний.}
% 	\topline
% 	\justifying
	
% \end{frame}

\begin{frame}{\\Понятие языка описания методов}
	\topline
	\justifying
	\begin{SCn}
    \scnheader{язык представления методов}
    \scnidtf{язык методов}
    \scnidtf{язык представления методов, соответствующих определенному классу методов}
    \scnidtf{язык (например, sc-язык) представления методов соответствующего класса методов}
    \scnsubset{язык}
    \scnidtf{язык программирования}
    \scnsuperset{язык представления методов обработки информации}
    \scnsuperset{язык представления методов решения задач во внешней среде кибернетических систем}
    \end{SCn}
\end{frame}

\begin{frame}{\\Описание процедурного метода}
	\topline
	\justifying
 
    \vspace{0.33cm}
    
    Рассмотрим более подробно понятие процедурной программы (процедурного метода). Каждая \textbf{\textit{процедурная программа}} представляет собой обобщенный план выполнения \textit{действий}, принадлежащих некоторому классу, то есть \textit{семантическую окрестность; ключевым sc-элементом\scnrolesign} является \textit{класс действий}, для элементов которого дополнительно детализируется процесс их выполнения.
    
    \vspace{0.33cm}
    
    Входным параметрам \textit{процедурной программы} в традиционном понимании соответствуют аргументы, соответствующие каждому \textit{действию} из \textit{класса действий}, описываемого данной \textit{процедурной программой}. При генерации на основе данной программы \textit{плана} выполнения конкретного \textit{действия} из данного класса эти аргументы принимают конкретные значения.
\end{frame}

\begin{frame}{\\Описание процедурного метода}
	\topline
	\justifying
    \vspace{0.33cm}
    
     Каждая \textit{процедурная программа} представляет собой систему описанных действий с дополнительным указанием для действия:
    \begin{textitemize}
        \item либо \textit{последовательности выполнения действий*} (передачи инициирования), когда условием выполнения (инициирования) действий является завершение выполнения одного из указанных или всех указанных действий;
        \item либо события в базе знаний или внешней среде, являющегося условием его инициирования;
        \item либо ситуации в базе знаний или внешней среде, являющейся условием его инициирования.
    \end{textitemize}
\end{frame}

%TODO
% \begin{frame}{Пример описания процедурного метода в базе знаний.}
% 	\topline
% 	\justifying
	
% \end{frame}

% \begin{frame}{Статические и динамические аргументы действий.}
% 	\topline
% 	\justifying
	
% \end{frame}

% \begin{frame}{Пример описания метода с зависимостями между поддействиями.}
% 	\topline
% 	\justifying
	
% \end{frame}

\begin{frame}{\\Понятие стратегии решения задач}
	\topline
	\justifying
	
	В литературе, посвященной построению решателей задач, встречается понятие \textbf{\textit{стратегии решения задач}}. Определим его как метаметод решения задач, обеспечивающий либо поиск одного релевантного известного метода, либо синтез целенаправленной последовательности акций применения в общем случае различных известных методов. 

    \begin{SCn}
    \scnheader{стратегия решения задач}
    \scnsubset{метод}
    \end{SCn}
\end{frame}

\begin{frame}{\\Понятие модели решения задач}
	\topline
	\justifying
	
	\vspace{10mm}
	По аналогии с понятием стратегии решения задач введем понятие \textbf{\textit{модели решения задач}}, которое будем трактовать как метаметод интерпретации соответствующего класса методов.
	
    \begin{SCn}
    \scnheader{модель решения задач}
    \scnsubset{метод}
    \scnidtf{метаметод}
    \scnidtf{абстрактная машина интерпретации соответствующего класса методов}
    \scnidtf{иерархическая система "микропрограмм"{}, обеспечивающих интерпретацию соответствующего класса методов}
    \scnsuperset{алгоритмическая модель решения задач}
    \scnsuperset{процедурная параллельная синхронная (асинхронная) модель решения задач}
    \end{SCn}
\end{frame}

\begin{frame}{\\Понятие модели решения задач}
	\topline
	\justifying
    \begin{SCn}
    \scnheader{модель решения задач}
    \scnsuperset{продукционная модель решения задач}
    \scnsuperset{функциональная модель решения задач}
    \scnsuperset{логическая модель решения задач}
    \scnsuperset{нейросетевая модель решения задач}
    \scnsuperset{генетическая модель решения задач}
    \scnheader{логическая модель решения задач}
    	\scnsuperset{четкая логическая модель решения задач}
    	\scnsuperset{нечеткая логическая модель решения задач}
    \end{SCn}
\end{frame}

\begin{frame}{\\Понятие модели решения задач}
	\topline
	\justifying
    \vspace{0.33cm}
    
    Каждая \textit{модель решения задач} задается:
    \begin{textitemize}
    	\item соответствующим классом методов решения задач, т.е. языком представления методов этого класса;
    	\item предметной областью этого класса методов; 
    	\item онтологией этого класса методов (т.е. денотационной семантикой языка представления этих методов);
    	\item операционной семантикой указанного класса методов.
    \end{textitemize}
    Модель решения задач ставит в соответствие некоторому классу методов синтаксис, денотационную и операционную семантику языка представления методов соответствующего класса.
\end{frame}

\begin{frame}{\\Понятие модели решения задач}
	\topline
	\justifying
    \begin{SCn}
    \scnheader{спецификация*}
    \scnsuperset{\textbf{модель решения задач}*}
    \scnheader{модель решения задач*}
    		\scneq{сужение отношения по первому домену (спецификация*; класс методов)*}
    		\scnidtf{спецификация \textit{класса методов}*}
    		\scnidtf{спецификация \textit{языка представления методов}*}
    \end{SCn}
\end{frame}

\begin{frame}{\\Понятие навыка}
	\topline
	\justifying
	
	Объединение \textit{метода} и его операционной семантики, то есть информации о том, каким образом должен интерпретироваться данный \textit{метод}, будем называть \textbf{\textit{навыком}}.
	
    \begin{SCn}
    \scnheader{навык}
    \scnidtf{умение}
    \scnidtf{объединение \textit{метода} с его исчерпывающей спецификацией -- \textit{полным представлением операционной семантики метода}}
    \scnidtf{метод, интерпретация (выполнение, использование) которого полностью может быть осуществлено данной кибернетической системой, в памяти которой хранится указанный метод}
    \end{SCn}
\end{frame}

\begin{frame}{\\Понятие навыка}
	\topline
	\justifying
        \begin{SCn}
        \scnheader{навык}
        \scnidtf{метод, который данная кибернетическая система умеет применять}
        \scnidtf{метод + метод его интерпретации}
        \scnidtf{умение решать соответствующий класс эквивалентных задач}
        \scnidtf{метод плюс его операционная семантика, описывающая то, как интерпретируется (выполняется, реализуется) этот метод, и являющаяся одновременно операционной семантикой соответствующей модели решения задач}
        \end{SCn}
\end{frame}

\begin{frame}{\\Классификация навыков}
	\topline
	\justifying
    \vspace{10mm}
    
Понятие \textit{навыка} является важнейшим понятием с точки зрения построения решателей задач, поскольку объединяет в себе не только декларативную и операционную части описания способа решения класса задач.

\textit{Навыки} могут быть \textit{пассивными навыками}, то есть такими \textit{навыками}, применение которых должно явно инициироваться каким-либо агентом, либо \textit{активными навыками}, которые инициируются самостоятельно при возникновении соответствующей ситуации в базе знаний. Для этого в состав \textit{активного навыка}, помимо \textit{метода} и его операционной семантики, включается также \textit{sc-агент}, который реагирует на появление соответствующей ситуации в базе знаний и инициирует интерпретацию \textit{метода} данного \textit{навыка}.
\end{frame}

\begin{frame}{Связь с базовым языком программирования для обработки баз знаний}
 	\topline
 	\justifying

 \begin{SCn}
	\scnheader{метод}
	\scnsuperset{scp-программа}
	
	\scnheader{действие}
	\scnsuperset{scp-оператор}
	
	\scnheader{язык описания методов}
	\scnhaselement{Язык SCP}
\end{SCn}
	
\end{frame}
