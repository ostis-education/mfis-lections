\title{Лекция 21\\Понятие решателя задач}
\author[]{Шункевич Д.В.}
\institute[]{Белорусский государственный университет информатики и радиоэлектроники}

\begin{frame}
	\titlepage
\end{frame}

\begin{frame}{\\Содержание лекции}
	\topline
	\justifying
	Современные подходы к решению задач в интеллектуальных системах, модели решения задач. Понятие решателя задач, основные отличительные особенности, характеристики, подходы к реализации. Основные недостатки современных подходов, принципы их устранения. Модель решателя задач как иерархической системы агентов, взаимодействующих посредством общей семантической памяти. Методика и средства разработки решателей задач.
\end{frame}

\begin{frame}{\\Современные подходы к решению задач}
	\topline
	\justifying
	\vspace*{\fill}\\
	\footnotesize{
		\begin{SCn}
			\scnheader{cовременные подходы к решению задач в компьютерных системах}
			\scnsuperset{Решение задач с использованием хранимых программ}
			\begin{scnindent}
				\scntext{\textit{пояснение}}{	В данном случае предполагается, что в системе заранее присутствует программа решения задачи заданного класса и решение сводится к поиску такой программы и интерпретации ее на заданных входных данных. К системам, ориентированным на такой подход к решению задач, относятся в том числе системы, использующие:
				\begin{textitemize}
					\item программы, написанные на языках программирования, относящихся как к императивной, так и к декларативной парадигме, в том числе логических и функциональных;
					\item реализации генетических алгоритмов;
					\item нейросетевые модели обработки знаний.
				\end{textitemize}}
			\end{scnindent}
		\end{SCn}
		Следует отметить, что даже в случае использования хранимой программы решение задачи далеко не всегда тривиально, поскольку, во-первых, требуется найти такую хранимую программу на основе некоторой спецификации, во-вторых, обеспечить ее интерпретацию.
	}
	
\end{frame}

\begin{frame}{\\Современные подходы к решению задач}
	\topline
	\justifying
	\vspace*{\fill}\\
	\footnotesize{
		\begin{SCn}
			\scnheader{cовременные подходы к решению задач в компьютерных системах}
			\scnsuperset{Решение задач в условиях, когда программа решения не известна}
			\begin{scnindent}
				\scntext{\textit{пояснение}}{	В этом случае предполагается, что в	системе необязательно присутствует готовая программа решения для класса задач, которому принадлежит некоторая сформулированная задача, подлежащая решению. В связи с этим необходимо применять дополнительные методы поиска путей решения задачи, не рассчитанные на какой-либо узкий класс задач (например, разбиение задачи на подзадачи, методы поиска решений в глубину и ширину, метод случайного поиска решения и метод проб и ошибок, метод деления пополам и др.), а также различные модели логического вывода: классические дедуктивные, индуктивные, абдуктивные; модели, основанные на нечетких логиках, логике умолчаний, темпоральной логике и многие другие.}
			\end{scnindent}
		\end{SCn}
	}
	
\end{frame}

\begin{frame}{\\Современные подходы к решению задач}
	\topline
	\justifying
	\vspace*{\fill}\\
	Однако при всем многообразии решаемых рассмотренными системами задач множество классов задач ограничивается имеющимся в системе набором жестко заданных приемов и алгоритмов решения задач, явно используемых при решении той или иной задачи. В то же время построение сложных систем, например, систем комплексной автоматизации, невозможно без обеспечения согласованного использования различных видов знаний и моделей решения задач в рамках одной системы при решении одной и той же \textit{комплексной задачи}. Кроме того, становится актуальной задача поддержки такой системы в состоянии, соответствующем текущему уровню развития технологий, дополнения ее более совершенными моделями и методами решения задач. При этом очевидно, что подобная реконфигурация системы должна осуществляться \underline{непосредственно в процессе эксплуатации} системы, а не требовать каждый раз, например, полной остановки всего производства или отдельных его частей.
\end{frame}

\begin{frame}{\\Решатель задач}
	\topline
	\justifying
	\begin{SCn}
		\scnheader{решатель задач}
		\scnidtf{компонент интеллектуальной системы, который отвечает за решение определенных задач с помощью алгоритмов и методов}
		\scntext{\textit{пояснение}}{Решатели задач могут включать в себя различные алгоритмы и методы, такие как математические моделирование, оптимизация, машинное обучение и другие подходы, которые позволяют эффективно решать задачи, связанные с управлением и контролем кибернетических систем.}
		
	\end{SCn}
\end{frame}

\begin{frame}{\\Решатель задач}
	\topline
	\justifying
	\vspace*{\fill}\\
	\begin{SCn}
		\scnheader{решатель задач ostis-систем}
		\scnidtf{совокупность всех навыков, которыми обладает ostis-система на текущий момент времени}
		\scntext{\textit{пояснение}}{Предлагаемый в рамках \textit{Технологии OSTIS} подход к построению решателей задач позволяет обеспечить их модифицируемость, что, в свою очередь, позволяет \textit{ostis-системе} при необходимости легко приобретать новые \textit{навыки}, модифицировать (совершенствовать) уже имеющиеся, и даже избавляться от некоторых навыков с целью повышения производительности системы. Таким образом, имеет смысл говорить не о жестко фиксированном решателе задач, который разрабатывается один раз при создании первой версии системы и далее не меняется, а о совокупности навыков, фиксированной в каждый текущий момент времени, но постоянно эволюционирующей.}
	\end{SCn}
\end{frame}

\begin{frame}{\\Решатель задач}
	\topline
	\justifying
	\vspace*{\fill}\\
	\begin{SCn}
		\scnheader{решатель задач ostis-систем}
		\scnsuperset{объединенный решатель задач ostis-системы}
		\begin{scnindent}
			\scnidtf{полный решатель задач ostis-системы}
			\scnidtf{интегрированный решатель задач ostis-системы}
			\scnidtf{решатель задач ostis-системы, реализующий все ее функциональные возможности, как основные, так и вспомогательные}
		\end{scnindent}
		\scnsuperset{гибридный решатель задач ostis-системы}
		\begin{scnindent}
			\scnidtf{решатель задач ostis-системы, реализующий две и более модели решения задач}
		\end{scnindent}
	\end{SCn}
\end{frame}

\begin{frame}{\\Решатель задач}
	\topline
	\justifying
	\vspace*{\fill}\\
	\small{
		\begin{SCn}
			\scnheader{объединенный решатель задач ostis-системы}
			\scntext{\textit{пояснение}}{В общем случае \textit{объединенный решатель задач ostis-системы}, решает задачи, связанные с:
				\begin{textitemize}
					\item обеспечением основных функциональных возможностей системы (например, решение явно сформулированных задач по требованию пользователя);
					\item обеспечением корректности и оптимизацией работы самой ostis-системы (перманентно на протяжении всего жизненного цикла ostis-системы);
					\item обеспечением повышения квалификации конечных пользователей и разработчиков ostis-системы;
					\item обеспечением автоматизации развития и управления развитием ostis-системы.
			\end{textitemize}}
		\end{SCn}
	}
\end{frame}

\begin{frame}{\\Гибридный решатель задач}
	\topline
	\justifying
	\vspace*{\fill}\\
	\small{
	\begin{SCn}
		\scnheader{гибридный решатель задач}
		\scntext{\textit{требования}}{
			\begin{textitemize}
				\item в каждый момент времени решатель должен обеспечивать решение задач из оговоренного класса за оговоренное время, при этом результат решения задачи должен удовлетворять некоторым известным требованиям. Другими словами, как и в случае современных компьютерных систем, корректность результатов решения задач на этапе разработки системы должна верифицироваться специальными методами, в том числе для этого могут быть использованы такие современные подходы, как unit-тестирование, тестирование методом «черного ящика» и др.
		\end{textitemize}}
	\end{SCn}
	}
\end{frame}

\begin{frame}{\\Гибридный решатель задач}
	\topline
	\justifying
	\vspace*{\fill}\\
	\footnotesize{
		\begin{SCn}
			\scnheader{гибридный решатель задач}
			\scntext{\textit{требования}}{
				\begin{textitemize}
					\item \textbf{гибридный решатель} должен обеспечивать возможность \textbf{согласованного использования различных моделей решения задач} при решении одной и той же комплексной задачи в случае необходимости;	
					\item решатель должен быть легко \textbf{модифицируемым}, т. е. трудоемкость внесения изменений в уже разработанный решатель должна быть минимальна.
					\item для того чтобы интеллектуальная система имела возможность анализировать и оптимизировать имеющийся решатель задач, интегрировать в его состав новые компоненты (в том числе самостоятельно), оценивать важность тех или иных компонентов и применимость их для решения той или иной задачи, спецификация решателя должна быть описана языком, понятным системе, например, при помощи тех же средств, что и обрабатываемые знания. Другими словами, интеллектуальная система и, соответственно, решатель должны обладать \textit{рефлексивностью}.
				\end{textitemize}}
		\end{SCn}
	}
\end{frame}

\begin{frame}{\\Недостатки моделей решателей задач}
	\topline
	\justifying
	\vspace*{\fill}\\
	Несмотря на то что в настоящее время существует большое число моделей решения задач, многие из которых реализованы и успешно используются на практике в различных системах, остается актуальной проблема низкой согласованности принципов, лежащих в основе реализации таких моделей, и отсутствия единой унифицированной основы для реализации и интеграции различных моделей, что приводит к тому, что:
	\begin{textitemize}
		\item затруднена возможность одновременного использования различных моделей решения задач в рамках одной системы при решении одной и той же комплексной задачи; практически невозможно комбинировать различные модели с целью решения задачи, для которой априори отсутствует алгоритм ее решения;
	\end{textitemize}
\end{frame}

\begin{frame}{\\Недостатки моделей решателей задач}
	\topline
	\justifying
	\begin{textitemize}
		\item практически невозможно использовать технические решения, реализованные в одной системе, в других системах, т. е. возможности использования компонентного подхода при построении решателей задач сильно ограничены. Как следствие, велико количество дублирований аналогичных решений в разных системах;
		\item фактически отсутствуют комплексные методики и средства построения решателей задач, которые бы обеспечивали возможность проектирования, реализации и отладки решателей различного вида.
	\end{textitemize}
\end{frame}

\begin{frame}{\\Недостатки моделей решателей задач}
	\topline
	\justifying
	Следствиями указанных проблем являются:
	\begin{textitemize}
		\item высокая трудоемкость разработки каждого решателя, увеличение сроков их разработки, а значит, и увеличение затрат на разработку и поддержку соответствующих интеллектуальных систем;
		\item высокая трудоемкость внесения изменений в уже разработанные решатели, т. е. отсутствует или сильно затруднена возможность дополнения уже разработанного решателя новыми компонентами и внесения изменений в уже существующие компоненты в процессе эксплуатации системы. Таким образом, высока трудоемкость поддержки разработанных решателей;
	\end{textitemize}
\end{frame}

\begin{frame}{\\Недостатки моделей решателей задач}
	\topline
	\justifying
	\vspace*{\fill}\\
	\begin{textitemize}
		\item высокий уровень профессиональных требований к разработчикам решателей задач, что обусловлено, в частности:
		\begin{textitemize}
			\item высокой сложностью существующих формализмов в области решения задач, рассчитанных на их интерпретацию компьютерной системой, а не человеком;
			\item отсутствием возможности рассматривать разрабатываемые решатели на разных уровнях детализации, выделения на каждом уровне достаточно независимых компонентов, что затрудняет процесс проектирования, тестирования и отладки таких решателей, а также снижает эффективность попыток объединения разработчиков решателей в коллективы по причине увеличения накладных расходов на согласование их деятельности;
			\item низким уровнем информационной поддержки разработчиков и автоматизации их деятельности.
		\end{textitemize}
	\end{textitemize}
\end{frame}

\begin{frame}{\\Подходы построения решателей задач}
	\topline
	\justifying
	\vspace*{\fill}\\
	\footnotesize{
		Для решения перечисленных проблем необходимо разработать комплекс моделей, методики и средств разработки \textit{гибридных решателей задач}. Исторически сложились два основных подхода к построению решателей задач \textit{интеллектуальных компьютерных систем.}\\
		\begin{SCn}
			\scnheader{подходы к построению решателей задач}
			\scnsuperset{Наличие в системе фиксированного решателя}
			\begin{scnindent}
				\scntext{\textit{пояснение}}{Предполагает наличие в системе фиксированного решателя (например, машины логического вывода), к которому впоследствии добавляется база знаний, наполнение которой определяется предметной областью, в которой должна работать система. Такие системы получили название "пустых" экспертных систем или "оболочек". Данный подход, как правило, использовался для разработки относительно несложных систем и в настоящее время не имеет широкого применения.}
			\end{scnindent}
		\end{SCn}
	}
\end{frame}

\begin{frame}{\\Подходы построения решателей задач}
	\topline
	\justifying
	\vspace*{\fill}\\
	\footnotesize{
		\begin{SCn}
			\scnheader{подходы к построению решателей задач}
			\scnsuperset{Наличие в системе программных средств доступа к информации}
			\begin{scnindent}
				\scntext{\textit{пояснение}}{В данный момент времени, широко используемый в настоящее время, предполагает наличие программных средств доступа к информации, хранящейся в некоторой базе, совместимых с различными популярными языками программирования. Данный подход широко используется, например, в системах, построенных на основе стандартов W3C, таких как \textbf{RDF}, \textbf{OWL}, \textbf{SPARQL}, а также графовых с.у.б.д., таких как \textbf{Neo4j}. Структура решателя, построенного на базе таких средств, определяется разработчиком в каждом конкретном случае и не фиксируется какими-либо стандартами. Такой подход обладает большей гибкостью, но отсутствие унификации в структуре и процессе разработки решателей приводит к отсутствию совместимости компонентов решателей, созданных разными разработчиками, большому количеству дублирований одних и тех же решений, повышению накладных расходов в процессе разработки и поддержки решателя.}
			\end{scnindent}
		\end{SCn}
	}
\end{frame}

\begin{frame}{\\Подходы построения решателей задач}
	\topline
	\justifying
	\vspace*{\fill}\\
	\small{
		Среди комплексных подходов к построению решателей задач, можно выделить проект \textbf{IACPaaS}, активно развивающийся в настоящее время. Целью данного проекта является разработка облачной платформы для построения на ее основе интеллектуальных сервисов различного назначения. В данном проекте активно используются библиотеки многократно используемых компонентов интеллектуальных систем. Конкретно для построения решателей задач, а также пользовательских интерфейсов таких систем используется многоагентный подход. Несмотря на близость некоторых технологических решений, реализуемых в проекте 	\textbf{IACPaaS} и в рамках \textbf{Технологии OSTIS}, основной целью указанного проекта является предоставление пользователю большого числа разнородных сервисов, выбор которых осуществляется самим пользователем, в то время как одним из ключевых принципов 	\textbf{Технологии OSTIS} является разработка общей формальной основы для интеграции различных моделей решения задач с целью их комбинирования при решении одной и той же комплексной задачи.	
	}
\end{frame}

\begin{frame}{\\Разработка гибридных решателей задач}
	\topline
	\justifying
	\vspace*{\fill}\\
	\footnotesize{
		\begin{SCn}
			\scnheader{разработка гибридных решателей задач на основе многоагентного подхода}
			\scntext{\textit{принципы}}{
			\begin{textitemize}
				\item в качестве основы для построения модели гибридного решателя задач предлагается использовать многоагентный подход. Данный подход позволяет обеспечить основу для построения параллельных асинхронных систем, имеющих распределенную архитектуру, повысить модифицируемость и производительность разработанных решателей; 
				\item процесс решения любой задачи предлагается декомпозировать на логически атомарные действия, что также позволит обеспечить совместимость и модифицируемость решателей;
				\item решатель предлагается рассматривать как иерархическую систему, состоящую из нескольких взаимосвязанных уровней. Такой подход позволяет обеспечить возможность проектирования, отладки и верификации компонентов на разных уровнях независимо от других уровней;
			\end{textitemize}}
		\end{SCn}
	}
\end{frame}

\begin{frame}{\\Разработка гибридных решателей задач}
	\topline
	\justifying
	\vspace*{\fill}\\
	\footnotesize{
		\begin{SCn}
			\scnheader{разработка гибридных решателей задач на основе многоагентного подхода}
			\scntext{\textit{принципы}}{
				\begin{textitemize}
					\item предлагается записывать \underline{всю} информацию о решателе и решаемых им задачах при помощи \textit{SC-кода} в той же базе знаний, что и собственно предметные знания системы. В общем случае такая информация включает:
					\begin{textitemize}
						\item спецификацию агентов, входящих в состав решателя;
						\item спецификацию методов, интерпретируемых агентами решателя;
						\item спецификацию всех информационных процессов, выполняемых агентами в семантической памяти, в том числе – конструкции, обеспечивающие синхронизацию выполнения параллельных процессов;
						\item спецификацию всех задач, на решение которых направлены указанные информационные процессы.
					\end{textitemize}
				\end{textitemize}}
		\end{SCn}
	}
\end{frame}

\begin{frame}{\\Преимущества данного подхода}
	\topline
	\justifying
	\vspace*{\fill}\\
	\footnotesize{
		\begin{SCn}
			\scnheader{разработка гибридных решателей задач на основе многоагентного подхода}
			\scntext{\textit{преимущества}}{
				\begin{textitemize}
					\item автономность (независимость) агентов в рамках такой системы, что позволяет локализовать изменения, вносимые в решатель при его эволюции, и снизить соответствующие трудозатраты, а также обеспечить устойчивость такой системы к отказам некоторых агентов;
					\item децентрализация обработки, т. е. отсутствие единого контролирующего центра, что также позволяет локализовать вносимые в решатель изменения;
					\item возможность параллельной работы разных информационных процессов, соответствующих как одному агенту, так и разным агентам, как следствие, – возможность распределенного решения задач.
					\item активность агентов и многоагентной системы в целом, дающая возможность при общении с такой системой не указывать явно способ решения поставленной задачи, а формулировать задачу в декларативном ключе.
				\end{textitemize}}
		\end{SCn}
	}
\end{frame}

\begin{frame}{Многоагентный подход}
	\topline
	\justifying
	\vspace*{\fill}\\
	\footnotesize{
		\begin{SCn}
			\scnheader{модель многоагентного подхода}
			\scnhaselement{модель агента}
			\begin{scnindent}
				\scntext{\textit{пояснение}}{модель самого агента, а также классификация таких агентов и набор понятий, характеризующих каждый агент в рамках системы. В настоящее время наиболее популярной является модель BDI (belief-desire-intention), в рамках которой предполагается описывать на соответствующих языках "убеждения", "желания" и "намерения" каждого агента системы}
			\end{scnindent}
				\scnhaselement{модель среды}
			\begin{scnindent}
				\scntext{\textit{пояснение}}{содержит информацию о среде в рамках которой находятся агенты, на события в которой они реагируют и в рамках которой могут осуществлять некоторые преобразования.}
			\end{scnindent}
				\scnhaselement{модель коммуникации агентов}
			\begin{scnindent}
				\scntext{\textit{пояснение}}{модель,в рамках которой уточняется язык взаимодействия агентов (структура и классификация сообщений) и способ передачи сообщений между агентами}
			\end{scnindent}
		\end{SCn}
	}
\end{frame}

\begin{frame}{Многоагентный подход}
	\topline
	\justifying
	\vspace*{\fill}\\
	\footnotesize{
		\begin{SCn}
			\scnheader{современные многоагентные систем}
			\scntext{\textit{недостатки}}{
				\begin{textitemize}
					\item жесткая ориентация большинства средств на модель BDI приводит к существенным накладным расходам, связанным с необходимостью выражения конкретной практической задачи в системе понятий BDI. В то же время ориентация на модель BDI неявно провоцирует искусственное разделение языков, описывающих собственно компоненты BDI и знания агента о внешней среде, что приводит к отсутствию унификации
					представления и, соответственно, дополнительным накладным расходам;
					\item большинство современных средств построения многоагентных систем ориентированы на представление знаний агента при помощи узкоспециализированных языков, зачастую не предназначенных для представления знаний в широком смысле. Речь при этом идет как о знаниях агента о себе самом, так и знаниях о внешней среде;
			\end{textitemize}}
		\end{SCn}
	}
\end{frame}

\begin{frame}{Многоагентный подход}
	\topline
	\justifying
	\vspace*{\fill}\\
	\scriptsize{
		\begin{SCn}
			\scnheader{современные многоагентные систем}
			\scntext{\textit{недостатки}}{
				\begin{textitemize}
					\item абсолютное большинство современных средств предполагает, что взаимодействие агентов осуществляется путем обмена сообщениями непосредственно от агента к агенту или посредством специальных коммуникационных центров. Такой подход обладает существенным недостатком, связанным с тем, что в этом случае каждый агент системы должен иметь достаточно полную информацию о других агентах в системе, что приводит к дополнительным затратам ресурсов;
					\item многие средства построения многоагентных систем построены таким образом, что логический уровень взаимодействия агентов жестко привязан к физическому уровню реализации многоагентной системы;
					\item в большинстве подходов среда, с которой взаимодействуют агенты, уточняется отдельно разработчиком для каждой многоагентной системы, что с одной стороны, расширяет возможности применения соответствующих средств, но, с другой стороны, приводит к существенным накладным расходам и несовместимости таких многоагентных систем.
			\end{textitemize}}
		\end{SCn}
	}
\end{frame}

\begin{frame}{Многоагентный подход}
	\topline
	\justifying
	\vspace*{\fill}\\
	\footnotesize{
		\begin{SCn}
			\scnheader{современные многоагентные систем}
			\scntext{\textit{пути решения проблем}}{
				\begin{textitemize}
					\item коммуникацию агентов предлагается осуществлять по принципу "доски объявлений", однако в отличие от классического подхода в роли сообщений выступают спецификации в общей семантической памяти выполняемых агентами действий (процессов), направленных на решение каких-либо задач, а в роли среды коммуникации выступает сама эта семантическая память;
					\item в роли внешней среды для агентов выступает та же семантическая память, в которой формулируются задачи и посредством которой осуществляется взаимодействие агентов. Такой подход обеспечивает унификацию среды для различных систем агентов, что, в свою очередь, обеспечивает их совместимость;
			\end{textitemize}}
		\end{SCn}
	}
\end{frame}

\begin{frame}{Многоагентный подход}
	\topline
	\justifying
	\vspace*{\fill}\\
	\footnotesize{
		\begin{SCn}
			\scnheader{современные многоагентные систем}
			\scntext{\textit{пути решения проблем}}{
				\begin{textitemize}
					\item спецификация каждого агента описывается средствами SC-кода в базе знаний;
					\item синхронизацию деятельности агентов предполагается осуществлять на уровне выполняемых ими процессов, направленных на решений тех или иных задач в семантической памяти. Таким образом, каждый агент трактуется как некий абстрактный процессор, способный решать задачи определенного класса;
					\item каждый информационный процесс в любой момент времени имеет ассоциативный доступ к необходимым фрагментам базы знаний, хранящейся в семантической памяти, за исключением фрагментов, заблокированных другими процессами. Таким образом, с одной стороны, исключается необходимость хранения каждым агентом информации о внешней среде, с другой стороны, каждый агент, как и в классических многоагентных системах, обладает только частью всей информации, необходимой для решения задачи;
			\end{textitemize}}
		\end{SCn}
	}
\end{frame}

\begin{frame}{Методика построения решателей задач}
	\topline
	\justifying
	\vspace*{\fill}\\
	\footnotesize{
		\begin{SCn}
			\scnheader{методика построения решателей задач}
			\scnsuperset{Формирование требований и спецификация решателя задач;}
			\scnsuperset{Формирование коллектива sc-агентов, входящих в состав разрабатываемого решателя;}
			\scnsuperset{Разработка алгоритмов атомарных sc-агентов;}
			\scnsuperset{Реализация scp-программ;}
			\scnsuperset{Верификация разработанных компонентов;}
			\scnsuperset{Отладка разработанных компонентов, исправление ошибок;}
		\end{SCn}
	}
\end{frame}

\begin{frame}{Методика построения решателей задач}
	\topline
	\justifying
	\vspace*{\fill}\\
	\footnotesize{
		\begin{SCn}
			\scnheader{формирование требований и спецификация решателя задач}
			\scntext{\textit{пояснение}}{
				На данном этапе необходимо:
				\begin{textitemize}		
					\item четко выделить задачи, решение которых должен обеспечивать решатель задач;	
					\item продумать предполагаемые способы их решения и на основе данного анализа определить место будущего решателя в общей иерархии решателей.
				\end{textitemize}
				Важность данного этапа заключается в том, что при правильной классификации существует вероятность того, что в составе библиотеки многократно используемых компонентов ostis-систем уже есть реализованный вариант требуемого решателя. В противном случае, у разработчика появляется возможность включить разработанный решатель в библиотеку многократно используемых компонентов ostis-систем для последующего использования. Данные факты обусловлены тем, что структура библиотеки многократно используемых компонентов решателей задач основана на семантической классификации таких решателей и, соответственно, их компонентов.}
		\end{SCn}
	}
\end{frame}

\begin{frame}{Методика построения решателей задач}
	\topline
	\justifying
	\vspace*{\fill}\\
	\small{
		\begin{SCn}
			\scnheader{Формирование коллектива sc-агентов, входящих в состав разрабатываемого решателя}
			\scntext{\textit{пояснение}}{
				В случае, когда найти в библиотеке готовый решатель, удовлетворяющий всем предъявляемым требованиям, не представляется возможным, необходимо выделить и специфицировать все компоненты такого решателя. Результатом данного этапа является перечень полностью специфицированных \textit{sc-агентов}, которые войдут в состав разрабатываемого решателя, с их иерархией вплоть до \textit{атомарных sc-агентов}. В рамках данного этапа очень важно проектировать коллектив агентов таким образом, чтобы максимально задействовать уже имеющиеся в библиотеке многократно используемые компоненты ostis-систем, а при отсутствии нужного компонента — иметь возможность включить его в библиотеку после реализации.}
		\end{SCn}
	}
\end{frame}

\begin{frame}{Методика построения решателей задач}
	\topline
	\justifying
	\vspace*{\fill}\\
	\small{
		\begin{SCn}
			\scnheader{Разработка алгоритмов атомарных sc-агентов}
			\scntext{\textit{пояснение}}{
				В рамках данного этапа необходимо продумать алгоритм работы каждого разрабатываемого \textit{атомарного sc-агента}. Разработка алгоритма подразумевает выделение в нем логически целостных фрагментов, которые могут быть реализованы как отдельные \textit{scp-программы}, в том числе выполняемые параллельно. Таким образом, появляется необходимость говорить не только о \textit{библиотеке многократно используемых абстрактных sc-агентов}, но и о \textit{библиотеке многократно используемых программ обработки sc-текстов} на различных языках программирования, в том числе \textit{библиотеке многократно используемых scp-программ}. Благодаря этому часть \textit{scp-программ}, реализующих алгоритм работы некоторого \textit{sc-агента}, может быть заимствована из соответствующей библиотеки.}
		\end{SCn}
	}
\end{frame}

\begin{frame}{Методика построения решателей задач}
	\topline
	\justifying
	\vspace*{\fill}\\
	\small{
		\begin{SCn}
			\scnheader{Реализация scp-программ}
			\scntext{\textit{пояснение}}{
				Конечным этапом непосредственно разработки является реализация специфицированных ранее scp-программ или при необходимости программ, реализуемых на уровне платформы.}
			
			\scnheader{Верификация разработанных компонентов}
			\scntext{\textit{пояснение}}{
				Верификация разработанных компонентов может осуществляться как вручную, так и с использованием специфицированных средств, входящих в состав системы автоматизации проектирования решателей задач ostis-систем.}
			
		\end{SCn}
	}
\end{frame}

\begin{frame}{Методика построения решателей задач}
	\topline
	\justifying
	\vspace*{\fill}\\
	\small{
		\begin{SCn}
			\scnheader{Отладка разработанных компонентов. Исправление ошибок}
			\scntext{\textit{пояснение}}{
				Этап отладки разработанных компонентов, в свою очередь, можно также условно разделить на более частные этапы:
				\begin{textitemize}				
					\item отладка отдельных scp-программ или программ, реализуемых на уровне платформы;
					\item отладка отдельных атомарных sc-агентов;
					\item отладка неатомарных sc-агентов, входящих в состав решателя задач;
					\item отладка всего решателя задач.
			\end{textitemize}}
			
		\end{SCn}
	}
\end{frame}