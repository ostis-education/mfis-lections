\title{Лекция 14\\Основные принципы разработки семантических моделей баз знаний}
\author[]{Шункевич Д.В.}
\institute[]{Белорусский государственный университет информатики и радиоэлектроники}

\begin{frame}
	\titlepage
\end{frame}

\begin{frame}{\\Содержание лекции}
	\topline
	\justifying
	Методика разработки баз знаний, основные этапы. Выделение иерархии предметных областей. Формирование онтологий.
\end{frame}

\begin{frame}{Методика разработки баз знаний}
	\topline
	\justifying 
	\begin{SCn}
	В современной литературе при анализе методов разработки баз знаний основное внимание уделяется анализу методологий разработки \textbf{онтологий}, которые являются основной частью современных баз знаний.
 
    Методология разработки онтологий представляет собой набор инструкций и руководств, описывающих процесс выполнения сложных процедур разработки онтологий. Она детализирует различные задачи, как они должны быть выполнены, в каком порядке и каким образом осуществлять документирование работы по созданию онтологий.
	\end{SCn}
\end{frame}

\begin{frame}{Основные этапы разработки баз знаний}
	\topline
	\justifying
	\begin{SCn}
        \vspace{8mm}
	\begin{textitemize}
		\item Формирование начальной структуры гибридной базы знаний.
            \item Выявление компонентов базы знаний, которые могут быть заимствованы из библиотеки многократно используемых компонентов баз знаний, и включение их в состав разрабатываемой базы знаний.
		\item Формирование проектных заданий на разработку недостающих фрагментов базы знаний и распределение заданий между разработчиками.
		\item Разработка и согласование фрагментов базы знаний, которые, в свою очередь, могут в дальнейшем быть включены в состав библиотеки многократно используемых компонентов баз знаний.
		\item Верификация и отладка базы знаний.
	\end{textitemize}
	\end{SCn}
\end{frame}

\begin{frame}{\\Иерархия предметных областей}
	\topline
	\justifying 
	\begin{SCn}
База знаний имеет иерархическую структуру, позволяющую рассматривать хранимые знания на различных уровнях детализации. Прежде всего это иерархия
предметных областей и соответствующих им онтологий.

Каждая из предметных областей описывает соответствующие классы объектов исследования с максимально возможной степенью детализации, определяемой
набором отношений и параметров, заданных на классах объектов исследования. На множестве предметных
областей, задано отношение \textbf{дочерняя предметная область*}, которое указывает направление наследования
свойств объектов исследования, рассматриваемых в разных предметных областях
	\end{SCn}
\end{frame}

\begin{frame}{\\Дочерняя предметная область}
	\topline
	\justifying 
	\begin{SCn}
                \scnheader{дочерняя предметная область*}
                \scniselement{неролевое отношение}
                \scniselement{бинарное отношение}
                \scnrelfrom{пояснение}{\normalfont{[}дочерняя предметная область* – бинарное ориентированное отношение, с помощью которого
                задается иерархия предметных областей путем перехода от менее детального к более детальному
                рассмотрению соответствующих исследуемых понятий.\normalfont{]}}
                \scnrelfrom{примечание}{\normalfont{[}Для любой предметной области все свойства ее объектов исследования наследуются всеми ее
                дочерними предметными областями*.\normalfont{]}}
	\end{SCn}
\end{frame}

\begin{frame}{\\Формирование онтологий}
	\topline
	\justifying 
	\begin{SCn}
                \scnheader{онтология}
                \scnsubset{метазнание}
                \scnidtf{семантическая спецификация любого знания, имеющего достаточно сложную структуру, любого целостного
                фрагмента базы знаний – предметной области, метода решения сложных задач некоторого класса, описания
                истории некоторого вида деятельности, описания области выполнения   некоторого множества действий
                (области решения задач), языка представления методов решения задач и т.д.}
                \scnrelfrom{пояснение}{\normalfont{[Основная цель построения онтологии – семантическое уточнение (пояснение, а в идеале – определение)
                такого семейства знаков, используемых в заданном знании, которых достаточно для понимания смысла
                всего специфицируемого знания. Как выясняется, количество знаков, смысл которых определяет смысл
                всего специфицируемого знания, не является большим.]}
}
	\end{SCn}
\end{frame}
