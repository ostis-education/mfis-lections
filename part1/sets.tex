\title{Лекция 5\\Представление в базе знаний множеств и операций над ними}   
\author[]{Шункевич Д.В.}
\institute[]{Белорусский государственный университет информатики и радиоэлектроники}

\begin{frame}
	\titlepage
\end{frame}

\begin{frame}{\\Содержание лекции}
	\topline
	\justifying
	Типология множеств, рефлексивное множество. Ориентированное множество, декартово произведение множеств. Атрибуты, кортежи и классические кортежи. Мощность, бесконечные и конечные множества (без представления в базе знаний). Отношения над множествами, равенство. Операции над канторовскими множествами. Операции над мультимножествами. Булеан множества, его мощность. Представление в базе знаний.
\end{frame}

\begin{frame}{\\Понятие множества}
	\topline
	\justifying
	Под \textbf{\textit{множеством}} понимается соединение в некое целое M определённых хорошо различимых предметов m нашего созерцания или нашего мышления (которые будут называться ``элементами'' множества M).\\
	\bigskip
	\textbf{\textit{множество}} -- мысленная сущность, которая связывает одну или несколько сущностей в целое.\\
	\bigskip
	Более формально \textbf{\textit{множество}} -- это абстрактный математический объект, состоящий из элементов. Связь множеств с их элементами задается бинарным ориентированным отношением \textbf{\textit{принадлежность*}}.\\
	\bigskip
	В Теории множеств \textbf{\textit{множество}} считается неопределяемым понятием, его можно только пояснить.
	\vspace{-3em}
\end{frame}

\begin{frame}{\\Задание множеств} %TODO Примеры
	\topline
	\justifying
	\textbf{\textit{множество}} может быть полностью задано следующими тремя способами:
	\begin{textitemize}
		\item путем перечисления (явного указания) всех его элементов (очевидно, что таким способом можно задать только конечное множество)
		\item с помощью определяющего высказывания, содержащего описание общего характеристического свойства, которым обладают все те и только те объекты, которые являются элементами (т.е. принадлежат) задаваемого множества.
		\item с помощью теоретико-множественных операций, позволяющих однозначно задавать новые множества на основе уже заданных (это операции объединения, пересечения, разности множеств и др.)
	\end{textitemize}
	\vspace{-1.5em}
\end{frame}

\begin{frame}{\\Принадлежность}
	\topline
	\justifying
	\begin{SCn}
		\scnheader{принадлежность*}
		\scnidtf{принадлежность элемента множеству*}
		\scniselement{бинарное отношение}
		\scniselement{ориентированное отношение}
	\end{SCn}
	
	\textbf{\textit{принадлежность*}} – это бинарное ориентированное отношение, каждая связка которого связывает множество с одним из его элементов.\\
	\bigskip
	Элементами отношения \textbf{\textit{принадлежность*}} в \textit{SC-коде} по умолчанию являются \textbf{\textit{базовые sc-дуги}} (позитивные постоянные константные sc-дуги принадлежности).
\end{frame}

\begin{frame}{\\Классификация множеств}
	\topline
	\justifying
	\begin{SCn}
	\scnheader{множество}
	\begin{scnrelfromset}{разбиение}
		\scnitem{конечное множество}
		\scnitem{бесконечное множество}
	\end{scnrelfromset}
	\begin{scnrelfromset}{разбиение}
		\scnitem{множество без кратных элементов}
		\scnitem{мультимножество}
	\end{scnrelfromset}
	\begin{scnrelfromset}{разбиение}
		\scnitem{кортеж}
		\scnitem{неориентированное множество}
	\end{scnrelfromset}
	\end{SCn}
\end{frame}

\begin{frame}{\\Конечность множеств}
	\topline
	\justifying
	\begin{SCn}
	\scnheader{конечное множество}
	\scnidtf{множество с конечным числом элементов}
	\end{SCn}

	\textbf{\textit{конечное множество}} – это \textit{множество}, количество элементов которого конечно, т.е. существует неотрицательное целое число \textit{k}, равное количеству элементов этого множества.
	\vspace{-\baselineskip}
	\begin{SCn}	
	\scnheader{бесконечное множество}
	\scnidtf{множество с бесконечным числом элементов}
	\begin{scnrelfromset}{разбиение}
		\scnitem{счетное множество}
		\scnitem{несчетное множество}
	\end{scnrelfromset}
	\end{SCn}
	\textbf{\textit{бесконечное множество}} -- это \textit{множество}, в котором для любого натурального числа \textit{n} найдётся конечное подмножество из \textit{n} элементов.

\end{frame}

\begin{frame}{\\Счетность множеств}
	\topline
	\justifying
	\textbf{\textit{счетное множество}} -- это \textit{бесконечное множество}, для которого существует \textit{взаимно однозначное соответствие} с натуральным рядом чисел.

	\begin{SCn}	
	\scnheader{несчетное множество}
	\scnidtf{континуальное множество}
	\end{SCn}

	\textbf{\textit{несчетное множество}} -- это \textit{бесконечное множество}, элементы которого невозможно пронумеровать натуральными числами.
\end{frame}

\begin{frame}{\\***}
	\topline
	\justifying

	\scnheader{множество без кратных элементов}
	\scnidtf{классическое множество}
	\scnidtf{множество, состоящее из разных элементов}
	\scntext{пояснение}{\textbf{\textit{множество без кратных элементов}} -- это \textit{множество}, для каждого элемента которого существует только одна пара принадлежности, выходящая из знака этого множества в указанный элемент.}
	
	\scnheader{мультимножество}
	\scnidtf{множество, имеющее кратные вхождения хотя бы одного элемента}
	\scnidtf{множество, по крайней мере один элемент которого входит в его состав многократно}
	\scntext{пояснение}{\textbf{\textit{мультимножество}} -- это \textit{множество}, для которого существует хотя бы одна кратная пара принадлежности, выходящая из знака этого множества.}
	
\end{frame}


\begin{comment}



\scnheader{кратность принадлежности}
\scnidtf{кратность принадлежности элемента}
\scnidtf{кратность вхождения элемента во множество}
\scniselement{параметр}
\scntext{пояснение}{\textbf{\textit{кратность принадлежности}} -- \textit{параметр}, значением которого являются числовые величины, показывающие сколько раз входит тот или иной элемент в рассматриваемое множество. Элементами данного параметра являются классы \textit{позитивных sc-дуг принадлежности}, связывающих данное множество с элементом, кратность вхождения которого в данное множество мы хотим задать. %TODO Пример

Таким образом, кратное вхождение элемента в мультимножество может быть задано как явным указанием \textit{позитивных sc-дуг принадлежности} этого элемента данному \textit{множеству}, так и «склеиванием» этих дуг в одну и включением ее в некоторый класс \textbf{\textit{кратности принадлежности}}.}
%\scnrelfrom{описание примера}{
%	\scnfilescg{figures/sd_sets/multiplicityOfMembership.png}
%}

%%

\scnheader{нечеткое множество}
\scntext{пояснение}{\textbf{\textit{нечеткое множество}} – это \textit{множество}, которое представляет собой совокупность элементов произвольной природы, относительно которых нельзя точно утверждать – обладают ли эти элементы некоторым характеристическим свойством, которое используется для задания этого нечеткого множества. Принадлежность элементов такому множеству указывается при помощи \textit{нечетких позитивных sc-дуг принадлежности}.}

\scnheader{четкое множество}
\scntext{пояснение}{\textbf{\textit{четкое множество}} – это \textit{множество}, принадлежность элементов которому достоверна и указывается при помощи \textit{четких позитивных sc-дуг принадлежности}.}

%%

\scnheader{семейство множеств}
\scnidtf{множество множеств}
\scnsuperset{класс классов}
\scntext{пояснение}{\textbf{\textit{семейство множеств}} – это \textit{множество}, элементами которого являются знаки множеств.}

\scnheader{нерефлексивное множество}
\scntext{пояснение}{\textbf{\textit{нерефлексивное множеств}} – это \textit{множество}, знак которого не является элементом этого множества}

\scnheader{рефлексивное множество}
\scntext{пояснение}{\textbf{\textit{рефлексивное множеств}} – это \textit{множество}, знак которого является элементом этого множества}

%%

\scnheader{множество}
\scnsuperset{пустое множество}
\scnsuperset{синглетон}
\scnsuperset{пара}
\scnsuperset{тройка}


\scnheader{пустое множество}
\scniselement{мощность множества}
\scntext{пояснение}{\textbf{\textit{пустое множество}} – это \textit{множество}, которому не принадлежит ни один элемент.}


\scnheader{синглетон}
\scniselement{мощность множества}
\scnidtf{множество мощности 1}
\scnidtf{одноэлементное множество}
\scnidtf{одномощное множество}
\scnidtf{множество, мощность которого равна 1}
\scnidtf{множество, имеющее мощность равную единице}
\scnidtf{синглетон из sc-элемента} 
\scnidtf{sc-синглеон}
\scnsubset{конечное множество}
\scntext{пояснение}{\textbf{\textit{синглетон}} – это \textit{множество}, состоящее из одного элемента.


\scnheader{пара}
\scniselement{мощность множества}
\scnidtf{множество мощности два}
\scnidtf{двухэлементное множество}
\scnidtf{двумощное множество}
\scnsubset{конечное множество}
\scntext{пояснение}{\textbf{\textit{пара}} – это \textit{множество}, состоящее из двух элементов.


\scnheader{тройка}
\scniselement{мощность множества}
\scnidtf{тройка}
\scnidtf{множество мощности три}
\scnsubset{конечное множество}
\scntext{пояснение}{\textbf{\textit{тройка}} – это \textit{множество}, состоящее из трех элементов.

\scnheader{мощность множества}
\scnidtf{кардинальное число}
\scnidtf{общее число вхождений элементов в заданное множество}
\scnidtf{класс эквивалентности, элементами которого являются знаки всех тех и только тех множеств, которые имеют одинаковую мощность}
\scnidtf{класс эквивалентности, соответствующий отношению быть парой множеств, имеющих одинаковую мощность (равномощность множеств)}
\scnidtf{величина мощности множеств}
\scnidtf{трансфинитное число}
\scnidtf{мощность по Кантору}
\scniselement{параметр}
\scntext{пояснение}{\textbf{\textit{мощность множества}} – это \textit{параметр}, элементами которых являются \textit{множества}, имеющие одинаковое количество элементов. Значением данного параметра является числовая величина, задающая количество элементов, входящих в данный класс множеств, т.е. по сути, количество \textit{позитивных sc-дуг принадлежности}, выходящих из данного \textit{множества}.

%Ориентированность

\scnheader{кортеж}%Ориентированное множество?
\scnidtf{вектор}
\scntext{пояснение}{\textbf{\textit{кортеж}} – это множество, представляющее собой упорядоченный набор элементов, т.е. такое множество, порядок элементов в котором имеет значение. Пары принадлежности элементов \textbf{\textit{кортежу}} могут дополнительно принадлежать каким-либо \textit{ролевым отношениям}, при этом, в рамках каждого \textbf{\textit{кортежа}} должен существовать хотя бы один элемент, роль которого дополнительно уточнена \textit{ролевым отношением}.}

%%Отношения на множествах

\scnheader{включение*}
\scnidtf{быть подмножеством*}
\scniselement{бинарное отношение}
\scniselement{ориентированное отношение}
\scniselement{транзитивное отношение}
\scnrelfrom{область определения}{множество}
\scnsuperset{строгое включение*}
\scntext{определение}{\textbf{\textit{включение*}} – это бинарное ориентированное отношение, каждая связка которого связывает два множества. Будем говорить, что \textit{Множество Si} \textbf{\textit{включает*}} в себя \textit{Множество Sj} в том и только том случае, если каждый элемент \textit{Множества Sj} является также и элементом \textit{Множества Si}.}
\scnrelfrom{описание примера}{
%	\scnfilescg{figures/sd_sets/inclusion.png}
}
\begin{scnindent}
\scntext{пояснение}Множество {Sj} включается во множество \textit{Si}.}
\end{scnindent}

\scnheader{строгое включение*}
\scnidtf{строгое включение множеств*}
\scnsubset{включение*}
\scniselement{бинарное отношение}
\scniselement{ориентированное отношение}
\scnrelfrom{область определения}{множество}
\scntext{определение}{\textbf{\textit{строгое включение*}} – это \textit{бинарное ориентированное отношение}, областью определения которого является семейство всевозможных множеств. Будем говорить, что \textit{Множество Si} \textbf{\textit{строго включает*}} в себя \textit{Множество Sj} в том и только том случае, если каждый элемент \textit{Множество Sj} является также и элементом \textit{Множество Si}, при этом существует хотя бы один элемент \textit{Множество Si}, не являющийся элементом \textit{Множество Sj}.}
\scnrelfrom{описание примера}{
%	\scnfilescg{figures/sd_sets/strictInclusion.png}
}
\begin{scnindent}
\scntext{пояснение}Множество \textit{Sj} строго включается во множество \textit{Si}.}
\end{scnindent}
\scnrelfrom{изображение}{
	\scnfileimage{\includegraphics[width=0.4\linewidth]{figures/sd_sets/inclusion2.png}}}

\scnheader{равенство множеств*}
\scniselement{бинарное отношение}
\scniselement{неориентированное отношение}
\scnidtf{быть равными множествами*}
\scntext{определение}{\textbf{\textit{равенство множеств}}* -- бинарное неориентированное отношение, выражающее отношение равенства множеств.
	
	Любые два множества являются равными множествами тогда и только тогда, когда первое является включением второго и второе является включением первого.}
\scnrelfrom{описание примера}{
%	\scnfilescg{figures/sd_sets/equalityOfSets.png}
}
\begin{scnindent}
\scntext{пояснение}Множество \textit{Si} равно множеству \textit{Sj}.}
\end{scnindent}


\scnheader{булеан*}
\scnidtf{булеан множества*}
\scnidtf{семейство всевозможных подмножеств заданного множества*}
\scniselement{бинарное отношение}
\scniselement{ориентированное отношение}
\scntext{определение}{\textbf{\textit{булеан*}} – это \textit{бинарное ориентированное отношение} между множеством и некоторым семейством множеств, каждое из которых является подмножеством первого множества.}
\scnrelfrom{описание примера}{
%	\scnfilescg{figures/sd_sets/boulean.png}
}

\scnheader{семейство подмножеств*}
\scnidtf{семейство подмножеств заданного множества*}
\scniselement{бинарное отношение}
\scniselement{ориентированное отношение}
\scnsuperset{булеан*}
\scntext{определение}{\textbf{\textit{семейство подмножеств*}} – это \textit{бинарное ориентированное отношение} между множеством и некоторым семейством множеств, каждое из которых является подмножеством первого множества.}
\scnrelfrom{описание примера}{
%	\scnfilescg{figures/sd_sets/familyOfSubsets.png}
}

%%Операции на множествах
\scnheader{объединение*}
\scnidtf{объединение множеств*}
\scniselement{квазибинарное отношение}
\scniselement{ориентированное отношение}
\scntext{определение}{\textbf{\textit{объединение*}} – это \textit{квазибинарное ориентированное отношение}, областью определения которого является семейство всевозможных множеств. Будем говорить, что \textit{Множество Si} является объединением \textit{Множество Sj} и \textit{Множество Sk} тогда и только тогда, когда любой элемент \textit{Множество Si} является элементом или \textit{Множество Sj} или \textit{Множество Sk}.}
\scnrelfrom{описание примера}{
%	\scnfilescg{figures/sd_sets/union.png}
}
\begin{scnindent}
\scntext{пояснение}{Множество \textit{Si} является объединением множеств \textit{Sj}, \textit{Sk} и \textit{Sm}.}
\end{scnindent}
%\scnrelfrom{изображение}{
%	\scnfileimage{\includegraphics[width=0.6\linewidth]{figures/sd_sets/union2.png}}}

\scnheader{разбиение*}
\scnidtf{разбиение  множества*}
\scnidtf{объединение попарно непересекающихся множеств*}
\scnidtf{декомпозиция множества*}
\scniselement{квазибинарное отношение}
\scniselement{ориентированное отношение}
\scniselement{отношение декомпозиции}
\scntext{определение}{\textbf{\textit{разбиение*}} – это \textit{квазибинарное ориентированное отношение}, областью определения которого является семейство всевозможных множеств. В результате разбиения множества получается множество попарно непересекающихся множеств, объединение которых есть исходное множество.\\
	Семейство множеств \{\textit{S1…, Sn}\} является разбиением множества \textit{Si} в том и только том случае, если:
	\begin{scnitemize}
		\item семейство \{\textit{S1…, Sn}\} является семейством \textit{попарно непересекающихся множеств};
		\item семейство \{\textit{S1…, Sn}\} является покрытием множества \textit{Si} (или другими словами, множество \textit{Si} является \textit{объединением} множеств, входящих в указанное выше семейство)
	\end{scnitemize}
}
\scnrelfrom{описание примера}{
%	\scnfilescg{figures/sd_sets/split.png}
}
\begin{scnindent}
\scntext{пояснение}{Множество \textit{Si} разбивается на множества \textit{Sj}, \textit{Sk} и \textit{Sm}.}
\end{scnindent}
%\scnrelfrom{изображение}{
%	\scnfileimage{\includegraphics[width=0.5\linewidth]{figures/sd_sets/split2.png}}}

\scnheader{пересечение*}
\scnidtf{пересечение множеств*}
\scniselement{квазибинарное отношение}
\scniselement{ориентированное отношение}
\scntext{определение}{\textbf{\textit{пересечение*}} – это операция над множествами, аргументами которой являются два или большее число множеств, а результатом является множество, элементами которого являются все те и только те сущности, которые одновременно принадлежат каждому множеству, которое входит в семейство аргументов этой операции.\\
	Будем говорить, что \textit{Множество Si} является пересечением \textit{Множество Sj} и \textit{Множество Sk} тогда и только тогда, когда любой элемент \textit{Множество Si} является элементом \textit{Множество Sj} и элементом \textit{Множество Sk}.}
\scnrelfrom{описание примера}{
%	\scnfilescg{figures/sd_sets/intersection.png}
}
\begin{scnindent}
\scntext{пояснение}{Множество \textit{Si} является результатом пересечения множеств \textit{Sj}, \textit{Sk} и \textit{Sm}.}
\end{scnindent}
%\scnrelfrom{изображение}{
%	\scnfileimage{\includegraphics[width=0.5\linewidth]{figures/sd_sets/intersection2.png}}}

\scnheader{разность множеств*}
\scniselement{бинарное отношение}
\scniselement{ориентированное отношение}
\scntext{определение}{\textbf{\textit{разность множеств*}} – это \textit{бинарное ориентированное отношение}, связывающее между собой \textit{ориентированную пару}, первым элементом которой является уменьшаемое множество, вторым -- вычитаемое множество, и множество, являющееся результатом вычитания вычитаемого из уменьшаемого, в которое входят все элементы первого множества, не входящие во второе множество.}
\scnrelfrom{описание примера}{
%	\scnfilescg{figures/sd_sets/setDifference.png}
}
\begin{scnindent}
\scntext{пояснение}{Множество \textit{Si} является результатом разности множеств \textit{Sj} и \textit{Sk}.}
\end{scnindent}
%\scnrelfrom{изображение}{\scnfileimage{\includegraphics[width=0.5\linewidth]{figures/sd_sets/setDifference2.png}}}

\scnheader{симметрическая разность множеств*}
\scniselement{бинарное отношение}
\scniselement{ориентированное отношение}
\scntext{определение}{\textbf{\textit{симметрическая разность множеств*}} – это \textit{бинарное ориентированное отношение}, связывающее между собой \textit{пару} множеств и множество, являющееся результатом симметрической разности элементов указанной пары. Будем называть \textit{Множество Si} результатом симметрической разности \textit{Множества Sj} и \textit{Множества Sk} тогда и только тогда, когда любой элемент \textit{Множества Si} является или элементом \textit{Множества Sj} или \textit{Множества Sk}, но не является элементом обоих множеств.}
%\scnrelfrom{описание примера}{
%	\scnfilescg{figures/sd_sets/symmetricDifferenceOfSets.png}
%	\scntext{пояснение}Множество \textit{Si} является результатом симметрической разности множеств \textit{Sj} и \textit{Sk}.}
%}
%\scnrelfrom{изображение}{
%	\scnfileimage{\includegraphics[width=0.5\linewidth]{figures/sd_sets/symmetricDifferenceOfSets2.png}}}

\scnheader{декартово произведение*}
\scnidtf{декартово произведение множеств*}
\scnidtf{прямое произведение множеств*}
\scniselement{бинарное отношение}
\scniselement{ориентированное отношение}
\scntext{определение}{\textbf{\textit{декартово произведение*}} – это \textit{бинарное ориентированное отношение} между \textit{ориентированной парой} множеств и \textit{множеством}, элементами которого являются всевозможные упорядоченные пары, первыми элементами которых являются элементы первого из указанных множеств, вторыми – элементы второго из указанных множеств.}
\scnrelfrom{описание примера}{
%	\scnfilescg{figures/sd_sets/cartesianMultiplication.png}}
\begin{scnindent}
%\scntext{пояснение}Множество \textit{Si} является результатом декартова произведения множеств \textit{Sj} и \textit{Sk}.}
\end{scnindent}

%Пересекающиеся множества
\scnheader{пара пересекающихся множеств*}
\scniselement{бинарное отношение}
\scniselement{неориентированное отношение}
\scntext{пояснение}{\textbf{\textit{пара пересекающихся множеств*}} – \textit{бинарное неориентированное отношение} между двумя \textit{множествами}, имеющими непустое \textit{пересечение*}.}
\scntext{определение}{\textbf{\textit{пара пересекающихся множеств*}} – \textit{бинарное неориентированное отношение} между двумя \textit{множествами}, имеющими, по крайней мере, один общий для этих двух множеств элемент.}
\scnrelfrom{описание примера}{
%	\scnfilescg{figures/sd_sets/pairOfIntersectingSets.png}
}
\begin{scnindent}
\scntext{пояснение}{Множество \textit{Si} и множество \textit{Sj} являются парой пересекающихся множеств.}
\end{scnindent}
%\scnrelfrom{изображение}{
%	\scnfileimage{\includegraphics[width=0.5\linewidth]{figures/sd_sets/pairOfIntersectingSets2.png}}}

\scnheader{попарно пересекающиеся множества*}
\scnidtf{семейство попарно пересекающихся множеств*}
\scnsuperset{пересекающиеся множества*}
\scniselement{отношение}
\scntext{определение}{\textbf{\textit{попарно пересекающиеся множества*}} – семейство множеств, каждая пара которых является парой пересекающихся множеств, т.е. каждая пара которых имеет хотя бы один общий элемент}
\scntext{примечание}{Не каждое \textit{семейство попарно пересекающихся множеств*} является \textit{семейством пересекающихся множеств*}, хотя обратное верно.}
\scnrelfrom{изображение}{
%	\scnfilescg{figures/sd_sets/pairwiseIntersectingSets.png}
}
\begin{scnindent}
%\scntext{пояснение}Множества \textit{Si}, \textit{Sj}, \textit{Sk} и \textit{Sl} являются попарно пересекающимися множествами.}
\end{scnindent}
\scnrelfrom{изображение}{
	\scnfileimage{\includegraphics[width=0.7\linewidth]{figures/sd_sets/pairwiseIntersectingSets2.png}}}

\scnheader{пересекающиеся множества*}
\scnidtf{семейство пересекающихся множеств*}
\scnidtf{быть семейством пересекающихся множеств*}
\scnidtf{семейство множеств, имеющих по крайней мере один элемент, являющийся общим для всех этих множеств*}
\scnsuperset{попарно пересекающиеся множества*}
\scntext{определение}{\textbf{\textit{пересекающиеся множества*}} – это семейство множеств, имеющих по крайней мере один общий для всех этих множеств элемент}
\scnrelfrom{описание примера}{
%	\scnfilescg{figures/sd_sets/intersectingSets.png}
}
\begin{scnindent}
%\scntext{пояснение}Множества \textit{Si}, \textit{Sj}, \textit{Sk} и \textit{Sl} являются пересекающимися множествами.}
\end{scnindent}

\scnheader{пара непересекающихся множеств*}
\scniselement{бинарное отношение}
\scniselement{неориентированное отношение}
\scntext{определение}{\textbf{\textit{пара непересекающихся множеств*}} – это \textit{бинарное неориентированное отношение} между \textit{множествами}, результатом \textit{пересечения*} которых есть пустое множество.}
\scnrelfrom{описание примера}{
%	\scnfilescg{figures/sd_sets/pairOfNonIntersectingSets.png}
}
\begin{scnindent}
%\scntext{пояснение}Множества \textit{Si} и \textit{Sj} являются парой непересекающихся множеств.}
\end{scnindent}
%\scnrelfrom{изображение}{
%	\scnfileimage{\includegraphics[width=0.5\linewidth]{figures/sd_sets/pairOfNonIntersectingSets2.png}}}

\scnheader{попарно непересекающиеся множества*}
\scnidtf{семейство попарно непересекающихся множеств*}
\scnsubset{непересекающиеся множества*}
\scntext{определение}{\textbf{\textit{попарно непересекающиеся множества*}} – семейство множеств, каждая пара которых является парой непересекающихся множеств, т.е. каждая пара которых не имеет ни одного общего элемента}
\scnrelfrom{изображение}{
%	\scnfilescg{figures/sd_sets/pairwiseNonIntersectingSets.png}
}
\begin{scnindent}
%\scntext{пояснение}Множества \textit{Si}, \textit{Sj}, \textit{Sk} и \textit{Sl} являются попарно непересекающимися множествами.}
\end{scnindent}

\scnheader{непересекающиеся множества*}
\scnidtf{семейство непересекающихся множеств*}
\scnidtf{быть семейством непересекающихся множеств*}
\scntext{определение}{\textbf{\textit{непересекающиеся множества*}} – это семейство множеств, не имеющих ни одного общего элемента для всех этих множеств}
\scnrelfrom{изображение}{
%	\scnfilescg{figures/sd_sets/nonIntersectingSets.png}
%	\scntext{пояснение}Множества \textit{Si}, \textit{Sj}, \textit{Sk} и \textit{Sl} являются непересекающимися множествами.}
}
\end{comment}