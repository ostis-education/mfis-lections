\title{Лекция 3\\Внешние информационные конструкции}   
\author[]{Шункевич Д.В.}
\institute[]{Белорусский государственный университет информатики и радиоэлектроники}

\begin{frame}
	\titlepage
\end{frame}

\begin{frame}{\\Содержание лекции}
	\topline
	\justifying
	Понятие внешней информационной конструкции, классификация. Понятие внутреннего файла базы знаний, классификация. Понятие идентификатора, типология идентификаторов. Примеры формализации идентификаторов.
\end{frame}

\begin{frame}{\\Понятие информация}
	\topline
	\justifying
	\begin{SCn}
		\scnheader{информация}
		\scnidtf{информационная конструкция}
		\scnidtf{информационная модель, состоящая из некоторого множества различных \textit{знаков}, обозначающих моделируемые (описываемые) \textit{сущности} любого вида и, в частности, \textit{знаков}, обозначающих различного вида связи между знаками описываемых \textit{сущностей} (такие связи чаще всего являются отражениями (моделями) связей между \textit{сущностями}, которые обозначаются связываемыми \textit{знаками})}
	\end{SCn}
	
\end{frame}

\begin{frame}{\\Информационная конструкция}
	\topline
	\justifying
	\begin{SCn}
		\scnheader{информационная конструкция}
		\scnidtf{конструкция (структура), содержащая некоторые сведения о некоторых сущностях. Форма представления ("изображения", "материализации"), форма структуризации (синтаксическая структура), а также \textit{смысл*} (денотационная семантика) \textit{информационных конструкций} могут быть самыми различными.}
	\end{SCn}
	
\end{frame}

\begin{frame}{\\Дискретная информационная конструкция}
	\topline
	\justifying
	\begin{SCn}
		\scnheader{дискретная информационная конструкция}
		\scnsubset{информационная конструкция}
		\scntext{\textit{пояснение}}{Каждая дискретная информационная конструкция — это информационная конструкция, смысл которой задается (1) множеством элементов (синтаксически атомарных фрагментов) этой информационной конструкции, (2) алфавитом этих элементов — семейством классов синтаксически эквивалентных элементов информационной конструкции, (3) принадлежностью каждого элемента информационной конструкции соответствующему классу синтаксически эквивалентных элементов информационной конструкции, (4) конфигурацией связей инцидентности между элементами информационной конструкции.}
	\end{SCn}
	
\end{frame}

\begin{frame}{\\Информационные конструкции}
	\topline
	\justifying
	\begin{SCn}
		\scnheader{информация}
		\begin{scnrelfromset}{\textit{разбиение}}
			\scnitem{внутренняя информационная конструкция}
			\begin{scnindent}
			\scnidtf{информационная конструкция, хранимая в памяти некоторой кибернетической системы, и непосредственно интерпретируемая (понимаемая) решателем задач этой системы}
			\end{scnindent}
			\scnitem{внешняя информационная конструкция}
			\begin{scnindent}
			\scnidtf{информационная конструкция, представленная на каком-либо внешнем носителе или в памяти другой кибернетической системы}
			\end{scnindent}
			\scnitem{файл}
			\begin{scnindent}
			\scnidtf{первичный электронный образ некоторой внешней информационной конструкции}
			\end{scnindent}
		\end{scnrelfromset}
	\end{SCn}
	
\end{frame}

\begin{frame}{\\Файл}
	\topline
	\justifying
	\vspace*{\fill}\\
	\small{
		\begin{SCn}
		\scnheader{файл}
		\scnidtf{sc-узел, обозначающий файл}
		\scnidtf{знак файла}
		\begin{scnrelfromset}{\textit{разбиение}}
			\scnitem{ея-файл}
			\begin{scnindent}
				\scnidtf{естественно-языковой файл}
			\end{scnindent}
			\scnitem{файл, являющийся текстом формального языка}
			\begin{scnindent}
				\scnsuperset{sc.g-файл}
				\scnsuperset{sc.n-файл}
				\scnsuperset{sc.s-файл}
			\end{scnindent}
			\scnitem{файл, отражающий процесс изменения sc.g-текста}
			\scnitem{графический файл}
			\scnitem{файл-изображение}
			\scnitem{видео-файл}
			\scnitem{аудио-файл}
		\end{scnrelfromset}
		
	\end{SCn}
}
	
\end{frame}

\begin{frame}{\\Файл}
	\topline
	\justifying
	\vspace*{\fill}\\
	\small{
	\begin{SCn}
		\scnheader{файл}
		\begin{scnrelfromset}{\textit{разбиение}}
			\scnitem{файл-экземпляр}
			\begin{scnindent}
				\scnidtf{файл, являющийся конкретным электронным документом или электронным образом конкретной внешней информационной конструкции}
			\end{scnindent}
			\scnitem{файл-образец}
			\begin{scnindent}
				\scnidtf{файл-класс \textit{ostis-системы}}
				\scnidtf{файл, являющийся одновременно также и знаком множества всевозможных экземпляров (копий) этого файла}
			\end{scnindent}
		\end{scnrelfromset}
		\begin{scnrelfromset}{\textit{разбиение}}
			\scnitem{внешний файл ostis-системы}
			\scnitem{\textbf{внутренний файл ostis-системы}}
		\end{scnrelfromset}
		
	\end{SCn}
}
\end{frame}


\begin{frame}{\\Файл}
	\topline
	\justifying
	\begin{SCn}
		\scnheader{файл}
		\scnidtf{информационная конструкция, которая не является sc-конструкцией и которая может храниться в файловой памяти ostis-системы}
		\scntext{\textit{примечание}}{файловая память \textit{ostis-системы}, хранящая различного рода \textit{информационные конструкции} (образы, модели), не являющиеся \textit{sc-конструкциями}, должна быть тесно связана с \textit{sc-памятью} этой же \textit{ostis-системы}. Как минимум, каждый файл \textit{ostis-системы} должен быть связан с тем \textit{sc-элементом}, которых является знаком этого файла (точнее, знаком синглетона, элементом которого является указанный файл)}
		
	\end{SCn}
	
\end{frame}

\begin{frame}{\\Файл}
	\topline
	\justifying
	\begin{SCn}
		\scnheader{файл}
		\scntext{\textit{примечание}}{Представление \textit{информационных конструкций} в виде файлов ориентировано на представление \textit{дискретных (!) информационных конструкций}. Поэтому "файловое" представление \textit{недискретных информационных конструкций} (например, различного рода сигналов) предполагает "дискретизацию" таких конструкций, т.е. преобразование их в \textit{дискретные}. Так преобразуются аудио-сигналы (в частности, речевые сообщения), изображения, видео-сигналы и др.}
		
	\end{SCn}
	
\end{frame}

\begin{frame}{\\Внутренний файл}
	\topline
	\justifying
	\begin{SCn}
		\scnheader{внутренний файл ostis-системы}
		\scniselement{синтаксически выделяемый класс sc-элементов в рамках SC-кода}
		\scniselement{семантически выделяемый класс sc-элементов в рамках SC-кода}
		\scntext{\textit{примечание}}{Данный класс sc-элементов, являющихся знаками файлов, хранимых в памяти \textit{ostis-систем}, в отличие от других синтаксически выделяемых классов \textit{sc-элементов}, представляет собой одновременно синтаксически и семантически выделяемый класс \textit{sc-элементов}. Это обусловлено (1) тем, что каждый экземпляр данного класса \textit{sc-элементов} является знаком конкретного файла, хранимого в памяти \textit{ostis-системы}, и (2) тем, что каждый файл, хранимый в памяти \textit{ostis-системы}, может и должен быть обозначен только таким \textit{sc-элементом}, который является экземпляром рассматриваемого класса \textit{sc-элементов}.}
	\end{SCn}
	
\end{frame}

\begin{frame}{\\Внутренний файл}
	\topline
	\justifying
	\begin{SCn}
		\scnheader{внутренний файл ostis-системы}
		\scntext{\textit{примечание}}{sc-узел может быть знаком файла, находящегося в памяти другой ostis-системы (не в той, в которой хранится этот sc-узел). Но в этом случае указанный sc-узел не будет принадлежать рассматриваемому
		классу sc-узлов.}
		\scntext{\textit{примечание}}{Следует отличать синтаксическую эквивалентность файлов, семантическую эквивалентность файлов и совпадение файлов (когда речь идет об одном и том же файле). Т.е. копия файла и один и тот же файл – это разные вещи}
	\end{SCn}
	
\end{frame}

\begin{frame}{\\Идентификатор}
	\topline
	\justifying
	\begin{SCn}
		\scnheader{sc-идентификатор}
		\scnidtf{строка символов или пиктограмма, взаимно однозначно представляющая соответствующий \textit{sc-элемент}, хранимый в \textit{sc-памяти}}
		\scnidtf{внешний идентификатор \textit{sc-элемента}}
		\begin{scnrelfromset}{\textit{разбиение}}
			\scnitem{простой sc-идентификатор}
			\begin{scnindent}
				\scnidtf{простой внешний идентификатор \textit{sc-элемента}}
			\end{scnindent}
			\scnitem{sc-выражение}
			\begin{scnindent}
				\scnidtf{сложный внешний идентификатор \textit{sc-элемента}, в состав которого входит один или несколько идентификаторов других \textit{sc-элементов}}
			\end{scnindent}
		\end{scnrelfromset}
	\end{SCn}
	
\end{frame}

\begin{frame}{\\Идентификатор}
	\topline
	\justifying
	\vspace*{\fill}\\
	\scriptsize{
	\begin{SCn}
		\scnheader{sc-идентификатор}
		\begin{scnrelfromset}{\textit{разбиение}}
			\scnitem{основной sc-идентификатор}
			\begin{scnindent}
				\scnidtf{основной \textit{sc-идентификатор} для носителей дополнительно указываемого языка общения (например, соответствующего естественного языка)}
				\scntext{\textit{примечание}}{основной \textit{sc-идентификатор} является уникальным (не имеет синонимов и омонимов) в рамках соответствующего естественного языка}
				\scnsuperset{основной международный \textit{sc-идентификатор}}
			\end{scnindent}
			\scnitem{неосновной sc-идентификатор}
			\begin{scnindent}
				\scnsuperset{(неосновной \textit{sc-идентификатор} $\cap$ пояснение)}
				\begin{scnindent}
					\scnidtf{неосновной \textit{sc-идентификатор}, являющийся одновременно и пояснением обозначаемой сущности}
					\scnsuperset{(неосновной \textit{sc-идентификатор} $\cap$ определение)}
					\begin{scnindent}
						\scnidtf{неосновной \textit{sc-идентификатор}, являющийся одновременно и определением обозначаемого понятия}
					\end{scnindent}
				\end{scnindent}
				\scnsuperset{неосновной часто используемый sc-идентификатор}
			\end{scnindent}
		\end{scnrelfromset}
	\end{SCn}
}
	
\end{frame}

\begin{frame}{\\Неосновной sc-идентификатор}
	\topline
	\justifying
	С помощью неосновных sc-идентификаторов указываются возможные \textit{синонимы*} соответствующего \textit{основного sc-идентификатора}, которые в частности, могут пояснять или даже определять обозначаемую сущность, указывает на важные свойства этой сущности.

	Для некоторых sc-элементов могут часто использоваться не только основные, но и неосновные sc-идентификаторы (особенно в неформальных текстах -- в пояснениях, примечаниях и т.п.). Явное выделение такого класса sc-идентификаторов позволяет упростить семантический анализ исходных текстов баз знаний.
\end{frame}

\begin{frame}{\\Основной sc-идентификатор}
	\topline
	\justifying
	\begin{SCn}
		\scnheader{основной sc-идентификатор}
		\scnsubset{файл-образец ostis-системы}
		\scnrelboth{\textit{семантическая эквивалентность}}{\scnfilelong{Все основные идентификаторы \textit{sc-элементов} в памяти \textit{ostis-системы} оформляются в виде копируемых фалов-образцов \textit{ostis-системы}.}}
		\begin{scnindent}
			\scntext{пояснение}{Копии основных \textit{sc-идентификаторов} входят в состав внешних текстов различных языков (\mbox{SCg-кода}, SCs-кода, SCn-кода), а также в различных падежах, склонения, спряжениях в состав файлов \textit{ostis-систем}.}
		\end{scnindent}
	\end{SCn}
	
\end{frame}

\begin{frame}{\\Построение основных sc-идентификатор}
	\topline
	\justifying
	В качестве \textit{основных sc-идентификаторов} могут использоваться также общепринятые международные условные обозначения некоторых сущностей, например, обозначения часто используемых функций (sin, cos, tg, log, и т.д.), единиц измерения, денежных единиц и многое другое. Формально каждый основной международный sc-идентификатор считается основным sc-идентификатором также и для каждого естественного языка, несмотря на то, что символы, используемые в основных международных sc-идентификаторах, могут не соответствовать алфавиту некоторых или даже всех естественных языков.
\end{frame}

\begin{frame}{\\Идентификатор}
	\topline
	\justifying
	\vspace*{\fill}\\
	\footnotesize{
	\begin{SCn}
		\scnheader{sc-идентификатор}
		\begin{scnrelfromset}{разбиение}
			\scnitem{строковый sc-идентификатор}
			\begin{scnindent}
				\scnidtf{\textit{sc-идентификатор}, представленный строкой символов, которая является именем обозначаемой сущности}
				\scnidtf{имя сущности, обозначаемой идентифицируемым \textit{sc-элементом}}
				\scnidtf{имя (термин, словосочетание), синонимичное соответствующему (идентифицируемому) \mbox{sc-элементу} и представленное в соответствующем алфавите символов}
			\end{scnindent}
			\scnitem{нестроковый sc-идентификатор}	
			\begin{scnindent}
				\scntext{пояснение}{В общем случае в качестве \textit{sc-идентификатора} некоторого \textit{sc-элемента} может выступать произвольный \textit{внутренний файл ostis-системы}, например, пиктограмма, условное обозначение или даже аудиофрагмент.}	
			\end{scnindent}
		\end{scnrelfromset}
		\scntext{примечание}{Введенные нами \textit{sc-идентификаторы} используются во всех внешних языках, близких SC-коду -- в SCg-коде, в SCs-коде и в SCn-коде.}
	\end{SCn}
}
	
\end{frame}


\begin{frame}{\\Строковый sc-идентификатор}
	\topline
	\justifying
	\small{
	\begin{SCn}
		\scnheader{строковый sc-идентификатор}
		\scnidtf{имя, приписываемое идентифицируемому \textit{sc-элементу}}
		\scnidtf{имя сущности, обозначаемой идентифицируемым \textit{sc-элементом}}
		\scnidtf{строка символов, синонимичная соответствующему идентифицируемому \textit{sc-элементу}}
		\scnsuperset{основной строковый sc-идентификатор}
		\begin{scnindent}
			\scnidtf{уникальное для каждого естественного языка внешнее имя, приписываемое идентифицируемому \textit{sc-элементу}}
			\scnsuperset{основной русскоязычный sc-идентификатор}
			\scnsuperset{системный sc-идентификатор}
			\scnsuperset{основной англоязычный sc-идентификатор}
			\scnsuperset{основной германоязычный sc-идентификатор}
			\scnsuperset{основной франкоязычный sc-идентификатор}
			\scnsuperset{основной италоязычный sc-идентификатор}
			\scnsuperset{основной китайскоязычный sc-идентификатор}
		\end{scnindent}
		\scnsuperset{системный sc-идентификатор}
	\end{SCn}
}
	
\end{frame}

\begin{frame}{\\Системный sc-идентификатор}
	\topline
	\justifying
	\begin{SCn}
		\scnheader{системный sc-идентификатор}
		\scnidtf{\textit{sc-идентификатор}, являющийся уникальным в рамках всей базы знаний \textit{Экосистемы OSTIS} (Глобальной базы знаний).}
		\scntext{пояснение}{Данный \textit{sc-идентификатор}, как правило, используется в исходных текстах базы знаний, при обмене сообщениями между \textit{ostis-системами}, а также для взаимодействия \textit{ostis-системы} с компонентами, реализованными с использованием средств, внешних с точки зрения \textit{Технологии OSTIS}, например, программ, написанных на традиционных языках программирования. Алфавит системных \textit{sc-идентификаторов} максимально упрощен для того, чтобы обеспечить удобство автоматической обработки таких \textit{sc-идентификаторов} с использованием современных технических средств, в частности, запрещены пробелы и различные специальные символы.}
	\end{SCn}
	
\end{frame}

\begin{frame}{\\Построение системных sc-идентификатор}
	\topline
	\justifying
	\vspace*{\fill}\\
	Символами, использующимися в \textit{системном sc-идентификаторе}, могут быть буквы латинского алфавита, цифры, знак нижнего подчеркивания и знак тире. Для обеспечения интернационализации рекомендуется формировать \textit{системные sc-идентификаторы} на основании основных англоязычных \textit{sc-идентификаторов}. Таким образом, наиболее целесообразно формировать \textit{системный sc-идентификатор} \textit{sc-элемента} из основного англоязычного путем замены всех символов, не входящих в описанный выше алфавит на символ нижнее подчеркивание (``\_''). Кроме того, заглавные буквы чаще всего заменяются на соответствующие строчные.
	
	Для именования sc-элементов, являющихся знаками \textit{ролевых отношений}, вместо знака ``\scnrolesign'' в \textit{системном sc-идентификаторе} используется приставка ``rrel'' и далее после нижнего подчеркивания записывается имя \textit{ролевого отношения}.
	
\end{frame}

\begin{frame}{\\Построение системных sc-идентификатор}
	\topline
	\justifying
	\vspace*{\fill}\\
	 Для именования sc-элементов, являющихся знаками \textit{неролевых отношений}, вместо знака ``*'' в \textit{системном sc-идентификаторе} используется приставка ``nrel'' и далее после нижнего подчеркивания записывается имя \textit{неролевого отношения}.\\
	
	Для именования sc-элементов, являющихся знаками классов \textit{понятий}~~в~~\textit{системном sc-идентификаторе} используется приставка ``concept'' и далее после нижнего подчеркивания записывается имя \textit{класса}.\\
	
	Для именования sc-элементов, являющихся знаками \textit{структур}~~в~~\textit{системном sc-идентификаторе} используется приставка ``struct'' и далее после нижнего подчеркивания записывается имя \textit{структуры}.
\end{frame}

\begin{frame}{\\Нетранслируемый sc-идентификатор}
	\topline
	\justifying
\begin{SCn}
	\scnheader{нетранслируемый sc-идентификатор}
	\scnsubset{системный sc-идентификатор}
	\scnidtf{\textit{sc-идентификато}, не представляемый в базе знаний \textit{ostis-системы}}
	\scnidtf{\textit{sc-идентификатор}, существующий только вне базы знаний \textit{ostis-системы}}	
\end{SCn}
	
\end{frame}

\begin{frame}{\\Нетранслируемый sc-идентификатор}
	\topline
	\justifying
	\vspace*{\fill}\\
	\footnotesize{
	Нетранслируемые sc-идентификаторы используются только в рамках исходных текстов \textit{баз знаний} (в том числе, \textit{sc.s-текстов}) и при обмене сообщениями между \textit{ostis-системами} в тех случаях, когда необходимо в нескольких фрагментах исходного текста \textit{базы знаний} или передаваемого сообщения использовать имя одного и того же \textit{sc-элемента}, но при этом указанный \textit{sc-элемент} не имеет \textit{системного sc-идентификатора} и вводить его нецелесообразно. Использование \textit{нетранслируемых sc-идентификаторов} позволяет повысить читабельность и структурированность исходных текстов \textit{баз знаний}, а также позволяет обратиться к одному и тому же неименуемому (в рамках базы знаний) \textit{sc-элементу} в разных файлах исходных текстов \textit{баз знаний} или в разных сообщениях, передаваемых между \textit{ostis-системами}. В качестве таких \textit{sc-элементов} часто выступают знаки \textit{структур} и \textit{связок}.

	Таким образом, \textit{нетранслируемые sc-идентификаторы} существуют только вне \textit{базы знаний ostis-системы} и при формировании базы знаний из исходных текстов или при погружении в базу знаний полученного сообщения соответствующий им \textit{внутренний файл ostis-системы} не создается.}
\end{frame}

\begin{frame}{\\Отличия идентификаторов}
	\topline
	\justifying
	\vspace*{\fill}\\
	\small{
		Системные идентификаторы отличаются от основных, во-первых, требованием к уникальности в рамках всей базы знаний \textit{Экосистемы OSTIS} (а, значит, независимостью от внешнего языка), а во-вторых, более простым алфавитом, удобным для автоматической обработки.
		
		\textit{Системные sc-идентификаторы} и \textit{нетранслируемые sc-идентификаторы} выполняют схожие функции, связанные с именованием \textit{sc-элементов} на уровне исходных текстов \textit{баз знаний} или передаваемых между \textit{ostis-системами} сообщений.
		
		Каждый \textit{системный sc-идентификатор} представляется в базе знаний в виде \textit{внутреннего файла ostis-системы} и связан с соответствующим \textit{sc-элементом} парой отношения \textit{системный \mbox{sc-идентификатор*}}. \textit{Нетранслируемые\\ sc-идентификаторы} не представляются в рамках \textit{базы знаний}, не имеют соответствующих \textit{внутренних файлов ostis-системы} и на уровне \textit{базы знаний} никак не связаны с идентифицируемыми ими \textit{sc-элементами}.
	}
\end{frame}

\begin{frame}{\\Простой sc - идентификатор}
	\topline
	\justifying
	\vspace*{\fill}\\
	Простой sc-идентификатор -- идентификатор sc-элемента, в состав которого идентификаторы других sc-элементов не входят и который не содержит \textit{транслируемую в SC-код} информацию об обозначаемой им сущности. 
	
	Простой строковый sc-идентификатор -- простой sc-идентификатор, представляющий собой строку (цепочку) символов, которая является именем (названием) той же сущности, что и идентифицируемый sc-элемент. Простые строковые sc-идентификаторы являются наиболее распространенным видом идентификаторов, приписываемых sc-элементам.
\end{frame}


\begin{frame}{\\Простой sc - идентификатор}
	\topline
	\justifying
	\vspace*{\fill}\\
	\small{
	\begin{SCn}
	\scnheader{простой строковый sc-идентификатор}
	\scnsuperset{системный sc-идентификатор}
	\scnsuperset{простой строковый идентификатор sc-переменной}
	\begin{scnindent}
		\scnhaselementrole{пример}{\scnfilelong{\_var1}}
	\end{scnindent}
	\scnsuperset{простой строковый sc-идентификатор неролевого отношения}
	\begin{scnindent}
		\scnidtf{простой строковый идентификатор sc-узла, являющегося знаком неролевого отношения}
		\scnhaselementrole{пример}{\scnfilelong{включение множеств*}}
	\end{scnindent}
	\scnsuperset{простой строковый sc-идентификатор ролевого отношения}
	\begin{scnindent}
		\scnidtf{простой строковый идентификатор sc-узла, являющегося знаком ролевого отношения}
		\scnhaselementrole{пример}{\scnfilelong{слагаемое\scnrolesign}}
	\end{scnindent}
\end{SCn}
}
\end{frame}


\begin{frame}{\\Простой sc - идентификатор}
	\topline
	\justifying
	\vspace*{\fill}\\
	\scriptsize{
	\begin{SCn}
	\scnsuperset{простой строковый sc-идентификатор класса классов}
	\begin{scnindent}
		\scnidtf{простой строковый идентификатор sc-узла, являющегося знаком класса классов}
		\scnhaselementrole{пример}{\scnfilelong{длина\scnsupergroupsign}}
	\end{scnindent}
	\scnsuperset{sc-идентификатор внешнего файла ostis-системы}
	\begin{scnindent}
		\scnidtf{URL-идентификатор}
		\scnhaselementrole{пример}{\scnfilelong{"file:///home/user/image1.png"{}}}
		\begin{scnindent}
			\scntext{примечание}{Данный sc-идентификатор описывает абсолютный путь к файлу под названием "image1.png"{}}
		\end{scnindent}	
		\scnhaselementrole{пример}{\scnfilelong{"file://image1.png"{}}}
		\begin{scnindent}
			\scntext{примечание}{Данный sc-идентификатор описывает относительный путь к файлу под названием "image1.png"{}}
		\end{scnindent}	
		
		\scnhaselementrole{пример}{\scnfilelong{"https://conf.ostis.net/content/image1.png"{}}}
		\begin{scnindent}
			\scntext{примечание}{Данный sc-идентификатор описывает путь к файлу под названием "image1.png"{}, расположенному на удаленном сервере.}
		\end{scnindent}	
	\end{scnindent}
	\end{SCn}
}
\end{frame}


\begin{frame}{\\Простой sc - идентификатор}
	\topline
	\justifying
	\vspace*{\fill}\\
	\small{
		\textit{sc-идентификаторы} внешних файлов ostis-систем предназначены для описания местоположения внешних файлов ostis-систем и представляют собой строку символов, которая строится в соответствии со стандартом URL, а затем берется в двойные кавычки. Кавычки нужны для однозначности определения того, где начинается и заканчивается данный sc-идентификатор, поскольку в общем случае в URL разрешены пробелы. Целесообразность этого обусловлена тем, что sc-идентификаторы данного типа часто используются в файлах исходных текстов баз знаний ostis-систем.
		
		\textit{sc-идентификаторы} внешних файлов ostis-систем с точки зрения Технологии OSTIS являются простыми строковыми sc-идентификаторами, хотя и могут содержать специальные символы, например "\%"{} или "/"{}. Это связано с тем, что указанные идентификаторы не несут в себе семантически значимой информации о свойствах самого sc-элемента, обозначаемого таким sc-идентификатором, а только информацию о его расположении в текущем состоянии внешнего мира ostis-системы.
	}
\end{frame}

\begin{frame}{\\Простой sc - идентификатор}
	\topline
	\justifying
	\vspace*{\fill}\\
	\textbf{Правила построения простых строковых sc-идентификаторов} включают в себя:
		\begin{textitemize}
			\item Алфавит символов, используемых в простых строковых\\ sc-идентификаторах;
			\item Префиксы и суффиксы, используемые в простых строковых\\ sc-идентификаторах;
			\item Разделители и ограничители, используемые в простых строковых sc-идентификаторах;
			\item Правила построения \textit{имен собственных} и \textit{имен нарицательных}, являющихся простыми строковыми sc-идентификаторами;
			\item Правила построения простых строковых sc-идентификаторов, определяемые различными классами идентифицируемых sc-элементов.
		\end{textitemize}
\end{frame}

\begin{frame}{\\Сложный sc - идентификатор}
	\topline
	\justifying
	\vspace*{\fill}\\
	
	\textbf{\textit{sc-выражение}} -- идентификатор, который не только обозначает соответствующую сущность, но также содержит информацию, представляющую собой по возможности однозначную спецификацию указанной сущности.
	
	Однозначную спецификацию сущности, которая является понятием, называют \uline{определением} этого понятия
\end{frame}

\begin{frame}{\\Сложный sc - идентификатор}
	\topline
	\justifying
	\vspace*{\fill}\\
	\scriptsize{
		\begin{SCn}
			\scnheader{sc-выражение}
			\scnidtf{имя соответствующей {\normalfont(}именуемой{\normalfont)} сущности построенное по принципу "та {\normalfont(}тот{\normalfont)}, которая {\normalfont(}который{\normalfont)} указываемым образом связана с другими указываемыми сущностями"{}}
			\scnidtf{выражение, идентифицирующее sc-элемент}
			\scnidtf{идентификатор sc-элемента, в состав которого входят другие идентификаторы и денотационная семантика которого точно определяется конкретным набором правил построения таких сложных (комплексных) идентификаторов, состоящих из определенным образом связанных между собой других идентификаторов}
			\scnidtf{сложный идентификатор, состоящий из других идентификаторов}
			\scnidtf{идентификатор, который представляет собой конструкцию, состоящую из нескольких других идентификаторов, а также из некоторых разделителей и ограничителей, и денотационная семантика которого \uline{однозначно} задается конфигурацией указанной конструкции}
			\scnidtf{сложный sc-идентификатор}
			\scnidtf{сложный (составной) внешний идентификатор sc-элемента}
			\scnidtf{выражение, идентифицирующее sc-элемент}
			\scnidtf{sc-идентификатор, в состав которого входит один или несколько простых sc-идентификаторов}
		\end{SCn}
	}
\end{frame}

\begin{frame}{\\Сложный sc - идентификатор}
	\topline
	\justifying
	\vspace*{\fill}\\
	\scriptsize{
		\textbf{\textit{sc-выражение}} разбивается на:
		\begin{textitemize}
			\item{\textit{sc-выражение неориентированного множества} -- sc-выражение, ограниченное фигурными скобками};
			\item{\textit{sc-выражение структуры} -- sc-выражение, обозначающее структуру, представленную на любом известном и легко определяемом языке (Русском, Английском, SCg-коде, SCs-коде, SCn-коде).
				\textit{sc-выражение структуры} обозначает структуру, содержащую sc-текст, семантически эквивалентный тому тексту (на некотором известном языке), который заключен в фигурные скобки. Чаще всего такой текст записывается на формальном языке, например, SCs-коде, и может быть автоматически однозначно интерпретирован. Возможна ситуация, когда указанный текст записан на менее неформальном языке, например, Русском, но в этом случае его автоматическая интерпретация значительно усложняется и в общем случае не всегда однозначна.
				В текущей реализации средств разработки исходных текстов баз знаний в соответствии с более старой версией правил простроения sc-выражений вместо фигурных скобок \textit{sc-выражение структуры} ограничивается квадратными скобками со звездочками ("[*"{} и "*]"{}).};
			\item{\textit{sc-выражение ориентированного множества} -- sc-выражение кортежа, ограничиваемое угловыми скобками и обозначающее упорядоченное (ориентированное) множество sc-элементов, порядок которых задаётся последовательностью перечисляемых их sc-идентификаторов.};	
		\end{textitemize}	
	}
\end{frame}
\begin{frame}{\\Сложный sc - идентификатор}
	\topline
	\justifying
	\vspace*{\fill}\\
	\scriptsize{
		\begin{textitemize}
			\item{\textit{sc-выражение внутреннего файла ostis-системы} -- sc-выражение, обозначающее \textit{внутренний файл ostis-системы}, визуальное представление (изображение) которого ограничивается квадратными скобками. sc-выражение внутреннего файла ostis-системы обозначает \textit{внутренний файл ostis-системы}, содержимое которого заключено в квадратные скобки, ограничивающие данное sc-выражение.
				Дополнительная спецификация \textit{внутреннего файла ostis-системы} легко осуществляется с помощью \textit{SC-кода}. Сюда входит \textit{язык}, на котором представлена \textit{информационная конструкция}, являющаяся содержимым \textit{файла}, формат кодирования, \textit{автор}* и многое другое.};
			\item{\textit{sc-выражение, обозначающее файл-образец ostis-системы} -- sc-выражение, ограниченное ограниченное квадратными скобками с восклицательными знаками.};
			\item{\textit{sc-выражение, построенное на основе бинарного отношения} -- sc-выражение, в состав которого входят \uline{либо} (1) \textit{sc-идентификатор}, обозначающий бинарное ориентированное отношение, и (2) в круглых скобках sc-идентификатор одного из элементов первого домена указанного бинарного ориентированного отношения, \uline{либо} (1) sc-идентификатор, обозначающий бинарное \uline{не}ориентированное отношение и (2) в круглых скобках sc-идентификатор одного из элементов области определения указанного бинарного неориентированного отношения. sc-выражение, построенное путём указания некоторого бинарного отношения (обычно функционального) и одного из его аргументов (в круглых скобках).};
		\end{textitemize}	
	}
\end{frame}
\begin{frame}{\\Сложный sc - идентификатор}
	\topline
	\justifying
	\vspace*{\fill}\\
	\scriptsize{
		\begin{textitemize}
			\item{\textit{sc-выражение, построенное на основе алгебраической операции} -- sc-выражение, ограниченное круглыми скобками и построенное путем указания \textit{sc-идентификаторов}, разделенных знаком алгебраической операции.};
			\item{\textit{sc-выражение, идентифицирующее sc-коннектор} -- sc-выражение, ограниченное круглыми скобками и идентифицирующее \textit{sc-коннектор}, инцидентный двум указанным sc-элементам и имеющий тип, задаваемый путем изображения соответствующего sc.s-коннектора.
				Для упрощения восприятия и обработки \textit{sc-выражений, идентифицирующих sc-коннектор} вводится следующее ограничение: первым и третьим компонентом такого sc-выражения может быть только \textit{простой sc-идентификатор}. В рамках sc.s-текстов внутри \textit{sc-выражений, идентифицирующих sc-коннектор} допускается также использование sc.s-модификаторов.}	
		\end{textitemize}	
	}
\end{frame}