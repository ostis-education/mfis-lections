
\title{Лекция 1\\Основные положения семантической технологии проектирования интеллектуальных компьютерных систем нового поколения \vspace{-2em}} 
\author[]{Шункевич Д.В.}
\institute[]{Белорусский государственный университет информатики и радиоэлектроники}

\begin{frame}
	\titlepage
\end{frame}

\begin{frame}{\\Содержание лекции}
	\topline
	\justifying
	\begin{itemize}
		\item[--] Понятие интеллекта, интеллектуальной системы, комплексной задачи.
		\item[--] Основные положения семантической технологии проектирования интеллектуальных компьютерных систем нового поколения.
		\item[--] Основные компоненты указанной технологии.
		\item[--] Понятие информационной конструкции, формального языка, знака, синтаксиса, семантики.
		\item[--] Понятие семантической памяти.
	\end{itemize}
\end{frame}

\begin{frame}{\\Что такое "интеллект"?}
	\vspace{10mm}
	Интеллектуальная система (ИС) -- система, которая может легко научиться решать новые задачи. То есть основной признак интеллектуальности -- обучаемость.\\ \vspace{5mm}
	Важно отличать:
	\begin{itemize}
		\item[--] способность обучаться более качественному решению задач определенного ограниченного класса (как это делают классические нейросетевые модели) \\
		\item[--] способность относительно легко переходить от решения задач одного класса к решению задач другого класса (с ограничениями или без них).
	\end{itemize}
\end{frame}

\begin{frame}{\\Что такое комплексная задача?}
	Комплексная задача - задача, для решения которой необходимо использовать различные виды заний и различные модели решения задач. \\ \vspace{3mm}
	Более того, невозможно заранее сказать, какой набор моделей потребуется для решения конкретной задачи. \\
	\vspace{5mm}
	Примеры комплексных задач:
	\begin{itemize}
		\item[--] понимание естественных языков, изображений, речевых сообщений \\
		\item[--] планирование поведения интеллектуальных роботов.
	\end{itemize}
\end{frame}

\begin{frame}{\\Гибридная интеллектуальная система}
	Гибридная интеллектуальная система (ГИС) -- интеллектуальная система, интегрирующая различные виды знаний и различные модели решения задач.\\
	Именно ГИС способны решать комплексные задачи. Однако у современных ГИС существует ряд недостатков:
	\begin{itemize}
		\item[--] монолитность 
		\item[--] решение конкретной задачи, а не различных классов задач
		\item[--] разработка требует колоссальных ресурсов
		\item[--] практически невозможно повторно использовать какие-либо компоненты систем для решения других задач, необходимо все делать заново
	\end{itemize}
\end{frame}

\begin{frame}{\\Что такое OSTIS?}
	\vspace{5mm}
	OSTIS (Open Semantic Technology for Intelligent Systems) – \textit{открытая} комплексная технология проектирования \textit{совместимых} интеллектуальных систем.\\
	Основные положения:
		\begin{itemize}
		\item[--] база знаний (БЗ) OSTIS может описывать любой вид знаний
		\item[--] решатель задач OSTIS основан на многоагентном подходе и позволяет легко комбинировать любые модели решения задач
		\item[--] интерфейс ostis-системы представляет собой подсистему со своей БЗ и решателем задач (также может быть описан с помощью SC-кода)
		\item[--] использование универсального способа представления (кодирования) информации, получившего название SC-код 
	\end{itemize}
\end{frame}


\begin{frame}{\\Достоинства OSTIS}
	\begin{itemize}
		\item[--] универсальность и унифицированность (единообразие) представления (любая информация может быть представлена одним и тем же способом)
		\item[--] удобство машинной обработки и восприятия человеком
		
	\end{itemize}
\end{frame}


\begin{frame}{\\}
	
\end{frame}