
\title{Лекция 1\\Основные положения семантической технологии проектирования интеллектуальных компьютерных систем нового поколения \vspace{-2em}} 
\author[]{Шункевич Д.В.}
\institute[]{Белорусский государственный университет информатики и радиоэлектроники}

\begin{frame}
	\titlepage
\end{frame}

\begin{frame}{\\Содержание лекции}
	\topline
	\justifying
	\begin{itemize}
		\item[--] Понятие интеллекта, интеллектуальной системы, комплексной задачи.
		\item[--] Основные положения семантической технологии проектирования интеллектуальных компьютерных систем нового поколения.
		\item[--] Основные компоненты указанной технологии.
		\item[--] Понятие информационной конструкции, формального языка, знака, синтаксиса, семантики.
		\item[--] Понятие семантической памяти.
	\end{itemize}
\end{frame}

\begin{frame}{\\Что такое "интеллект"?}
	\vspace{10mm}
	Интеллектуальная система (ИС) -- система, которая может легко научиться решать новые задачи. То есть основной признак интеллектуальности -- обучаемость.\\ \vspace{5mm}
	Важно отличать:
	\begin{itemize}
		\item[--] способность обучаться более качественному решению задач определенного ограниченного класса (как это делают классические нейросетевые модели) \\
		\item[--] способность относительно легко переходить от решения задач одного класса к решению задач другого класса (с ограничениями или без них).
	\end{itemize}
\end{frame}

\begin{frame}{\\Что такое комплексная задача?}
	Комплексная задача - задача, для решения которой необходимо использовать различные виды заний и различные модели решения задач. \\ \vspace{3mm}
	Более того, невозможно заранее сказать, какой набор моделей потребуется для решения конкретной задачи. \\
	\vspace{5mm}
	Примеры комплексных задач:
	\begin{itemize}
		\item[--] понимание естественных языков, изображений, речевых сообщений \\
		\item[--] планирование поведения интеллектуальных роботов.
	\end{itemize}
\end{frame}

\begin{frame}{\\Гибридная интеллектуальная система}
	Гибридная интеллектуальная система (ГИС) -- интеллектуальная система, интегрирующая различные виды знаний и различные модели решения задач.\\
	Именно ГИС способны решать комплексные задачи. Однако у современных ГИС существует ряд недостатков:
	\begin{itemize}
		\item[--] монолитность 
		\item[--] решение конкретной задачи, а не различных классов задач
		\item[--] разработка требует колоссальных ресурсов
		\item[--] практически невозможно повторно использовать какие-либо компоненты систем для решения других задач, необходимо все делать заново
	\end{itemize}
\end{frame}

\begin{frame}{\\Что такое OSTIS?}
	\vspace{5mm}
	OSTIS (Open Semantic Technology for Intelligent Systems) – \textit{открытая} комплексная технология проектирования \textit{совместимых} интеллектуальных систем.\\
	Основные положения:
		\begin{itemize}
		\item[--] база знаний (БЗ) OSTIS может описывать любой вид знаний
		\item[--] решатель задач OSTIS основан на многоагентном подходе и позволяет легко комбинировать любые модели решения задач
		\item[--] интерфейс ostis-системы представляет собой подсистему со своей БЗ и решателем задач (также может быть описан с помощью SC-кода)
		\item[--] использование универсального способа представления (кодирования) информации, получившего название SC-код 
	\end{itemize}
\end{frame}

\begin{frame}{\\Достоинства OSTIS}	
	\vspace{5mm}
	\begin{itemize}
		\item[--] Унифицированность (единообразие) представления (любая информация может быть представлена одним и тем же способом)
		\item[--] Удобство машинной обработки и восприятия человеком
		\item[--] Любые знания и модели решения задач могут быть легко интегрированы в ostis-систему (по принципу plug \& play). Систему всегда можно переобучить для решения другой задачи
		\item[--] Компоненты ostis-систем универсальны и совместимы друг с другом. Можно создавать библиотеку компонентов и использовать их повторно (что позволяет сократить время разработки новых компонентов на 40-60\%)
		\item[--] Система описывается с помощью SC-кода, поэтому она может анализировать себя, искать в себе ошибки и оптимизировать собственную работу, то есть обладает рефлексивностью
	\end{itemize}
\end{frame}

\begin{frame}{\\Достоинства OSTIS}
	\vspace{10mm}
	\begin{itemize}
		\item[--] Платформенная независимость. Разработка ostis-системы осуществляется независимо от операционной системы и архитектуры компьютера. Платформа может быть реализована как в программном варианте (виртуальная машина), так и в аппаратном варианте
		\item[--] Благодаря особому многоагентному подходу ostis-системы ориентированы на параллельную обработку информации
		\item[--] Ostis-система может включать в себя компоненты, разработанные на базе OSTIS, а также объединяться с любыми другими системами и интегрировать другие компоненты через специальный протокол обмена информацией (JSON) и/или программный интерфейс (API)
		\item[--] Производительность ostis-системы не хуже традиционной системы, а в иногда может оказаться лучше за счет параллельной обработки. При переходе на семантические компьютеры производительность будет еще выше.
	\end{itemize}
\end{frame}
  
\begin{frame}{\\Важно понимать}
	\vspace{10mm}
	\begin{itemize}
		\item [--] OSTIS -- это не конкретная интеллектуальная система, а технология разработки интеллектуальных систем, каждая из которых, в свою очередь, будет решать задачи определенного класса
		\item[--] Ключевые преимущества OSTIS заключаются не в принципиально новых функциональных возможностях разрабатываемых систем (большинство функций ostis-систем можно реализовать с помощью традиционных средств), а в том, насколько легко модифицировать и развивать разрабатываемые системы, адаптировать их к новым задачам, а также насколько эффективно можно накапливать и использовать полученные компоненты при разработке новых систем, сокращая при этом время и трудоемкость их разработки
		\item[--] OSTIS -- это способ решения проблемы совместимости, одной из важнейших проблем современных технологий
	\end{itemize}
\end{frame}	

\begin{frame}{\\Архитектура OSTIS})
	
\end{frame}